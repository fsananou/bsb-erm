% !TEX TS-program = xelatex
% Compile with XeLaTeX or LuaLaTeX (because of fontspec)

\documentclass[11pt,oneside]{article}

% ------------------------------------------------
% Layout
% ------------------------------------------------
\usepackage[a4paper,margin=1in]{geometry}

% ------------------------------------------------
% Fonts (XeLaTeX/LuaLaTeX)
% ------------------------------------------------
\usepackage{fontspec}
\setmainfont{Carlito}

% ------------------------------------------------
% Typography & spacing
% ------------------------------------------------
\usepackage{microtype}
\usepackage{setspace}
\onehalfspacing
\setlength{\parindent}{0pt}
\setlength{\parskip}{0.6em}

% ------------------------------------------------
% Math
% ------------------------------------------------
\usepackage{amsmath,amssymb,mathtools}

% ------------------------------------------------
% Tables & lists
% ------------------------------------------------
\usepackage{booktabs}
\usepackage{enumitem}

% ------------------------------------------------
% Hyperlinks
% ------------------------------------------------
\usepackage[colorlinks=true,linkcolor=black,urlcolor=black,citecolor=black]{hyperref}

% ------------------------------------------------
% Title
% ------------------------------------------------
\title{\textbf{Risk Measurement Exercises: Value-at-Risk (VaR) and Expected Shortfall (ES)}}
\author{}
\date{}

% ------------------------------------------------
% Convenience macros (optional)
% ------------------------------------------------
\newcommand{\VaR}{\mathrm{VaR}}
\newcommand{\ES}{\mathrm{ES}}
\newcommand{\Loss}{L}
\newcommand{\R}{R}
\newcommand{\Normal}{\mathcal{N}}

\begin{document}
\maketitle

\section*{Context}
This document contains exercises designed to build intuition and technical skill for
\textbf{Value-at-Risk (VaR)} and \textbf{Expected Shortfall (ES)}. No prior knowledge of ERM is required.

\bigskip
\hrule
\bigskip

% =================================================
\section{Exercise 1 — Interpreting Value-at-Risk (Conceptual)}
A trading desk reports a \textbf{1-day VaR at 99\% confidence} of \textbf{€10 million}.

\begin{enumerate}[label=\alph*)]
  \item Explain in plain language what this number means.
  \item Out of 100 trading days, on how many days should losses exceed €10 million?
  \item Does this measure tell you how large losses can be beyond €10 million?
  \item Can the loss exceed €10 million? Explain.
\end{enumerate}

\vspace{0.5em}
\textit{Answer in words; no calculations required.}

% =================================================
\section{Exercise 2 — Mathematical Definition of VaR}
Let $\Loss$ be a real-valued random variable denoting the \textbf{loss} of a portfolio over one day
(so larger $\Loss$ means worse outcomes).

\begin{enumerate}[label=\alph*)]
  \item Write a mathematical definition of $\VaR_{\alpha}(\Loss)$ for a confidence level $\alpha \in (0,1)$ using the CDF of $\Loss$.
  \item If $q_\alpha$ denotes the $\alpha$-quantile of $\Loss$, explain the relationship between $\VaR_{\alpha}(\Loss)$ and $q_\alpha$.
  \item In one sentence, explain why $\VaR$ is a \textbf{quantile} and not an \textbf{expected value}.
\end{enumerate}

% =================================================
\section{Exercise 3 — Parametric VaR under Normal Returns (with Numbers)}
Assume daily \textbf{returns} $\R$ are normally distributed:
\[
\R \sim \Normal(\mu,\sigma^2),\quad \mu = 0.10\% = 0.001,\quad \sigma = 2.0\% = 0.02.
\]
Assume the portfolio value is €100 million and define daily \textbf{loss} as
\[
\Loss = -V\R,\quad V = 100{,}000{,}000.
\]

\begin{enumerate}[label=\alph*)]
  \item Compute the \textbf{95\%} 1-day $\VaR$ in euros (use $z_{0.95}=1.645$).
  \item Compute the \textbf{99\%} 1-day $\VaR$ in euros (use $z_{0.99}=2.326$).
  \item Briefly explain why increasing the confidence level increases $\VaR$.
\end{enumerate}

% =================================================
\section{Exercise 4 — Expected Shortfall under Normality (Formula + Calculation)}
Continue with Exercise 3 and assume $\Loss$ is normal.

For a standard normal $Z \sim \Normal(0,1)$, you may use:
\[
\phi(z)=\frac{1}{\sqrt{2\pi}}e^{-z^2/2}, \qquad \ES_{\alpha}(Z)=\frac{\phi(z_\alpha)}{1-\alpha},
\]
where $z_\alpha$ is the $\alpha$-quantile of $Z$ (e.g., $z_{0.95}=1.645$, $z_{0.99}=2.326$).

\begin{enumerate}[label=\alph*)]
  \item Write $\ES_{\alpha}(\Loss)$ in terms of $\mu,\sigma,V$ and $z_\alpha$.
  \item Compute $\ES_{0.95}(\Loss)$ in euros (use $\phi(1.645)\approx 0.103$).
  \item Compute $\ES_{0.99}(\Loss)$ in euros (use $\phi(2.326)\approx 0.026$).
  \item Explain (intuitively) why $\ES$ captures information that $\VaR$ does not.
\end{enumerate}

% =================================================
\section{Exercise 5 — Historical VaR (Order Statistics)}
You observe the following daily portfolio returns (in \%) over 10 days:

\begin{center}
\begin{tabular}{cccccccccc}
\toprule
-3.2 & -2.5 & -1.8 & -1.2 & -0.9 & -0.4 & 0.1 & 0.4 & 0.7 & 1.0 \\
\bottomrule
\end{tabular}
\end{center}

Assume all observations are equally likely.

\begin{enumerate}[label=\alph*)]
  \item Sort the returns from worst to best.
  \item Compute the \textbf{95\% historical VaR} for \textbf{losses} if portfolio value is €50 million.
  \item State clearly which observation(s) determine the 95\% historical VaR in this small sample.
\end{enumerate}

% =================================================
\section{Exercise 6 — Historical Expected Shortfall (Tail Average)}
Use the same return sample as in Exercise 5 and portfolio value €50 million.

\begin{enumerate}[label=\alph*)]
  \item Identify the tail observations used to compute $\ES_{0.95}$ (based on your VaR rule from Exercise 5).
  \item Compute the \textbf{95\% historical Expected Shortfall} in euros.
  \item Explain in one sentence how ES reacts if the single worst return becomes even worse (e.g. from -3.2\% to -8\%).
\end{enumerate}

% =================================================
\section{Exercise 7 — Scaling VaR Over Time (Square-root-of-time)}
Assume daily returns are i.i.d. normal with mean $0$ and volatility $\sigma_{1d}=1.5\%$.
Let $V=€200$ million.

\begin{enumerate}[label=\alph*)]
  \item Compute the 1-day 99\% VaR in euros (use $z_{0.99}=2.326$).
  \item Under the square-root-of-time rule, compute the 10-day 99\% VaR:
  \[
  \sigma_{10d}=\sigma_{1d}\sqrt{10}.
  \]
  \item Give two reasons why this scaling can fail in practice (no calculations required).
\end{enumerate}

% =================================================
\section{Exercise 8 — Normal vs Fat Tails (Concept + Implications)}
Two portfolios have the same mean and variance of returns.
Portfolio A has normally distributed returns.
Portfolio B has Student-$t$ returns with low degrees of freedom (fat tails).

\begin{enumerate}[label=\alph*)]
  \item Which portfolio is likely to have higher $\VaR_{0.99}$? Why?
  \item Which portfolio is likely to have higher $\ES_{0.99}$? Why?
  \item Explain why ES is generally more sensitive to tail thickness than VaR.
\end{enumerate}

% =================================================
\section{Exercise 9 — Properties of Risk Measures (Coherence)}
A risk measure $\rho(\cdot)$ is called \textbf{coherent} if it satisfies:
monotonicity, translation invariance, positive homogeneity, and subadditivity.

\begin{enumerate}[label=\alph*)]
  \item State each of the four coherence properties in words.
  \item Explain what \textbf{subadditivity} means for diversification.
  \item True/False (justify briefly): VaR is always subadditive.
  \item True/False (justify briefly): Expected Shortfall is subadditive.
\end{enumerate}

% =================================================
\section{Exercise 10 — VaR vs ES in a Simple Discrete Example (Math)}
Consider a loss variable $\Loss$ (in €m) with the following distribution:

\begin{center}
\begin{tabular}{lccccc}
\toprule
Loss $\Loss$ (in €m) & 0 & 1 & 2 & 10 & 50 \\
Probability & 0.90 & 0.05 & 0.03 & 0.015 & 0.005 \\
\bottomrule
\end{tabular}
\end{center}

\begin{enumerate}[label=\alph*)]
  \item Compute $\VaR_{0.95}(\Loss)$.
  \item Compute $\VaR_{0.99}(\Loss)$.
  \item Compute $\ES_{0.95}(\Loss)$.
  \item Compute $\ES_{0.99}(\Loss)$.
  \item Compare VaR and ES: which one reacts more strongly to the rare €50m loss, and why?
\end{enumerate}

\bigskip
\hrule
\bigskip

\section*{Key Takeaways}
\begin{itemize}
  \item $\VaR$ is a \textbf{quantile}: it answers “how bad can it get, most of the time?”
  \item $\ES$ is a \textbf{tail average}: it answers “how bad is it when it goes really wrong?”
  \item Two portfolios can have the same $\VaR$ but very different tail risk (captured by $\ES$).
\end{itemize}

\end{document}
