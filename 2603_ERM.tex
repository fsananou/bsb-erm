\documentclass[aspectratio=169, 11pt]{beamer} % smaller for 10 pt
\usepackage[english]{babel}
\usepackage[utf8x]{inputenc}
%
% Choose how your presentation looks.
% For more themes, color themes and font themes, see:
% http://deic.uab.es/~iblanes/beamer_gallery/index_by_theme.html
%
\definecolor{myblue}{RGB}{15,75,205}
\definecolor{myred}{RGB}{164,16,52}
\usecolortheme[named=myblue]{structure} 

%\geometry{paperwidth=500pt, paperheight=375pt}
%\usepackage{kpfonts}
%
\mode<presentation>
{
    \usetheme{default}      %metropolis or try Darmstadt, Madrid, Warsaw, ...
    \usecolortheme{seahorse} % or try albatross, beaver, crane, ...
    \usefonttheme{default}  % or try serif, structurebold, ...
    \setbeamertemplate{navigation symbols}{}
    \addtobeamertemplate{navigation symbols}{}{%
        \usebeamerfont{footline}%
        \usebeamercolor[fg]{footline}%
        \hspace{100cm}%
        \insertframenumber $|$\inserttotalframenumber
    }
}
\usepackage{hyperref}
\usepackage{pgf, tikz, xcolor} % pour mes dessins latex
\usetikzlibrary{
	arrows,%
	calc,%
	shapes.geometric,%
	shapes.misc,%
	shapes.symbols,%
	shapes.arrows,%
	automata,%
	through,%
	positioning,%
	scopes,%
	decorations.shapes,%
	decorations.text,%
	decorations.pathmorphing,%
	shadows.blur}
% exemple de petit box
%\begin{tikzpicture}
%    \node [fill=cyan, rounded corners=5pt] {Hello};
%\end{tikzpicture}
\usepackage{fancybox}
\usepackage{hyperref}
\usepackage{natbib} %for advanced citations
\usepackage{pifont} %for the dings
\usepackage{ragged2e}
\usepackage{mathtools}

\usepackage{fontspec}
\setmainfont{Carlito}


% 
\usepackage{graphicx}
\newcommand\sbullet[1][.5]{\mathbin{\vcenter{\hbox{\scalebox{#1}{$\bullet$}}}}} % for bullets small, tiny, etc.
% https://tex.stackexchange.com/questions/389238/is-there-a-black-dot-symbol-that-i-can-use


%
\title{Entreprise risk management}
%\subtitle{Focus on the bond market}
\author{ \\ 
Foly Ananou, PhD \\
    %\footnotesize{Ph.D. Student, Economics} \\
    %\scriptsize{Université de Montréal and CIREQ} \\
   % \vspace{.25 cm}
   % \normalsize{ \textbf{Supervisor} \\
    %    Immo Schott} \\
   % \footnotesize{Assistant Professor, Economics} \\
   % \scriptsize{Université de Montréal and CIREQ}
    }
%\date{\today}
\date{}
%\titlegraphic{\includegraphics[height=1.5cm, width=\textwidth]{logo.jpg}}

%-------------------------------------------------------
\begin{document}
\begin{frame}[plain] % noframenumbering for beginining at 0
  % Ph.D. Project
    %\vspace{.5 cm}
  %\centering {\tiny \textsc{{\color{myblue} \Ovalbox{\textbf{Oral Exam}}}} }
  \centering
  %\includegraphics[height=1cm, width=0.6\textwidth]{logo.jpg}
    \vspace{1cm}

  \maketitle
\end{frame}

% Uncomment these lines for an automatically generated outline.
%\begin{frame}{Outline}
%  \setbeamertemplate{section in toc}[sections numbered]
%  \tableofcontents%[hideallsubsections]
%\end{frame}

%\AtBeginSection[]
%{
%    \begin{frame}[plain]{Next in the outline}
%        \setbeamertemplate{section in toc}[sections numbered]
%		\tableofcontents[currentsection]%, hidecurrentsubsection, hideothersubsections] %, pausesubsections
%	\end{frame}
%}


\section{Course objectives}

\subsection{Course objectives}
\begin{frame}{Course objectives}
    \begin{itemize}
        \item[$\bullet$] Master ERM fundamentals : understand the principles, frameworks, 
        and benefits of ERM
\medskip

\item[$\bullet$] Analyse and measure risk: learn to identify risk, assess and quantify risks
 using qualitative and quantitative tools 
\medskip

        \item[$\bullet$] Understand the implementation of ERM and best practices
        \medskip


   \end{itemize}
\end{frame}

%*****************************************
\subsection{Chap1}
\begin{frame}{}

\centering
\color{blue} \Ovalbox{\textbf{ERM: fundamentals}}

\end{frame}
%*****************************************
\subsection{Chap1}
\begin{frame}{ Understanding risk: preview}

    \begin{itemize}
        \item [$\bullet$] Let consider the following situations: 
                               \medskip
            \begin{itemize}
                \item[$\bullet$] You are in vacation in Cote d'Ivoire and decide to have an 
                outdoor gathering with friends at the beach. $\rightarrowtail $ What could 
                possibly go wrong ? How likely could this happen ? if it does happen, will it 
                be very harmful ? What can you do to prevent the event from happening or reduce 
                the impact ? 
                       \medskip

 \item[$\bullet$] You got it involve in a car crash $\rightarrowtail $ Is it necessary 
 someone fault or is it just one of those things that happen in life ? 
            \end{itemize}


    \end{itemize}
\end{frame}

%*****************************************
\subsection{Chap1}
\begin{frame}{Understanduing risk: a definition}
    \begin{itemize}
        \item [$\bullet$] Berstein (1996) defines risk as the uncertainty around the 
        outcome of a decision, process or event which can be negative (losses) or 
        positive (benefits or opportunities).

         \centering
  \includegraphics[height=6cm]{risk_bell_curve.png}
    \vspace{1cm} 

    \end{itemize}
\end{frame}

\subsection{Chap1}
\begin{frame}{Understanduing risk: key characteristics of risk}
    \begin{itemize}
        \item [$\bullet$] Uncertainty : outcomes are not guaranteed or predictable
         \medskip 

                 \item [$\bullet$] Dual nature: risk is not only about threats or losses,
                  it can also includes opportunities          \medskip 

                                  \item [$\bullet$] Measurable dimensions: 
    \begin{itemize}
        \item [$\bullet$] \Ovalbox{\textbf{exposure}} $\rightarrowtail $ what is at stake ? 
         \medskip 
          \item [$\bullet$] \Ovalbox{\textbf{probability}} $\rightarrowtail $ how likely is it possible ? 
         \medskip 
          \item [$\bullet$] \Ovalbox{\textbf{severity}} $\rightarrowtail $ how bad (or good) coud it get ? 
         \medskip 
          \item [$\bullet$] \Ovalbox{\textbf{time}} $\rightarrowtail $ how long ? 
         \medskip 
          \item [$\bullet$] \Ovalbox{\textbf{correlation}} $\rightarrowtail $ can it escalate or drive
          other risks ? 
         \medskip 
    \end{itemize}

    \end{itemize}
\end{frame}

\subsection{Chap1}
\begin{frame}{Understanduing risk: categorization}
    \begin{itemize}

                 \item [$\bullet$] Can you identify the risk type in the following sceenarios ? 
                 (let guess together) 
    \begin{itemize}
        \item [$\bullet$] A large international bank experiences a sophisticated ransomware attack
        that encrypts critical customers data and shuts down online banking services for almost 2W. 
         \medskip 
          \item [$\bullet$] The government announces overnight that all cryptoccurrencies transactions
          are banned taking effect immediately, forcing fintech companies to halt operatiions 
          immediately.  
         \medskip 
          \item [$\bullet$] An investment firm relies heavily on a proprietary risk model to allocate 
          capital. During a period of market stress, the model significantly underestimates correlations 
          across asset classes, leading to losses far exceeding management’s stated risk appetite.
         \medskip 
          \item [$\bullet$] An extreme weather event severely damages key production facilities of an
          energy company, forcing a prolonged shutdown. At the same time, regulators announce tighter
          environmental standards, increasing future compliance and investment costs.
    \end{itemize}

    \end{itemize}
\end{frame}

\subsection{Chap1}
\begin{frame}{Understanduing risk: categorization}
    \begin{itemize}

                 \item [$\bullet$] There is no "one-size-fit-all" in terms of risk category :
                 it's dependent on the analysis framework.  
    \begin{itemize}
        \item [$\bullet$] Market risk can be related to financial markets, for firms it could 
        relate to their ability to compete in a given (chosen) market(s)
          \medskip 
          \item [$\bullet$] Business risk can indicate the full scope of risks faced by a firm or just a subset
          (specific) risk related to the type of business the firm is envolved in (insurance risk for example)
         \medskip 
          \item [$\bullet$] Credit risk can include or exclude risk of changes in observed market 
          credit spread -- with some hint also to liquidity risk.  
    \end{itemize}

    \end{itemize}
\end{frame}

\subsection{Chap1}
\begin{frame}{Understanduing risk: financial risk (credit, liquidity and interest-rate)}


    \begin{itemize}

                 \item [$\bullet$] A bank lends money to firms and households, and its funding is essentially based on customers deposits.  
                 \begin{itemize}
        \item [$\bullet$] In what situations does the bank lose money ?
        \begin{itemize}
            \item [$\square$] Total or partial ? Why ? 
            \item [$\square$] Will the losses occur immediately or over time ? 
        \end{itemize} 
          \medskip 
          \item [$\bullet$] In what situation does the bank run out of cash ? 
           \begin{itemize}
            \item [$\square$] Does that mean the bank is insolvent ?  
            \item [$\square$] Can the bank fail in that situation ? 
        \end{itemize} 
           \medskip 
          \item [$\bullet$] In what situation does the bank profit fluctuate even there is nothing wrong on its customers side ? 
          \begin{itemize}
            \item [$\square$] The bank makes 20-year fixed-rate loans at 2\%. It funds itself with deposits whose interest rate can change every year.
What happen if interest rate rise to 5\% ? 
        \end{itemize} 
    \end{itemize}
    \end{itemize}
\end{frame}

\subsection{Chap1}
\subsection{Chap1}
\begin{frame}{Understanduing risk: financial risk (credit, liquidity and interest-rate)}
\begin{columns}[T,onlytextwidth]
\begin{column}{0.52\textwidth}
\begin{itemize}
    \item [$\bullet$] Commonly used and simplest measures are the Value-at-risk (VaR) and Expected Shortfall (ES)


\begin{itemize}
  \item VaR summarizes potential financial losses within a firm, portfolio,
  or position over a specific time frame. It reflects the worst expected loss
  under a given time horizon \ldots
  \begin{itemize}
    \item \ldots given a certain confidence level.
  \end{itemize}
  \medskip
  \item Expected shortfall extends VaR by capturing tail risk beyond the VaR cutoff.
  \item \href{run:var_es_exercices.pdf}{VaR and ES exercices}
\end{itemize}
\end{itemize}
\end{column}

\begin{column}{0.48\textwidth}
\centering
\includegraphics[width=\linewidth]{var.png} % <-- change path
\end{column}
\end{columns}

\end{frame}


\end{document}

