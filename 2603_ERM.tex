\documentclass[aspectratio=169, 10pt]{beamer} % smaller for 10 pt
\usepackage[english]{babel}
\usepackage[utf8x]{inputenc}

\setbeamertemplate{itemize item}{$\bullet$}
\setbeamertemplate{itemize subitem}{$\bullet$}
\setbeamertemplate{itemize subsubitem}{$\bullet$}

%
% Choose how your presentation looks.
% For more themes, color themes and font themes, see:
% http://deic.uab.es/~iblanes/beamer_gallery/index_by_theme.html
%
\definecolor{myblue}{RGB}{15,75,205}
\definecolor{myred}{RGB}{164,16,52}
\usecolortheme[named=myblue]{structure} 

%\geometry{paperwidth=500pt, paperheight=375pt}
%\usepackage{kpfonts}
%
\mode<presentation>
{
    \usetheme{default}      %metropolis or try Darmstadt, Madrid, Warsaw, ...
    \usecolortheme{seahorse} % or try albatross, beaver, crane, ...
    \usefonttheme{default}  % or try serif, structurebold, ...
    \setbeamertemplate{navigation symbols}{}
    \addtobeamertemplate{navigation symbols}{}{%
        \usebeamerfont{footline}%
        \usebeamercolor[fg]{footline}%
        \hspace{100cm}%
        \insertframenumber $|$\inserttotalframenumber
    }
}
\usepackage{hyperref}
\hypersetup{colorlinks=blue, linkcolor=blue, urlcolor=blue}
\usepackage{pgf, tikz, xcolor} % pour mes dessins latex
\usetikzlibrary{
	arrows,%
	calc,%
	shapes.geometric,%
	shapes.misc,%
	shapes.symbols,%
	shapes.arrows,%
	automata,%
	through,%
	positioning,%
	scopes,%
	decorations.shapes,%
	decorations.text,%
	decorations.pathmorphing,%
	shadows.blur}
% exemple de petit box
%\begin{tikzpicture}
%    \node [fill=cyan, rounded corners=5pt] {Hello};
%\end{tikzpicture}
\usepackage{fancybox}
\usepackage{hyperref}
\usepackage{natbib} %for advanced citations
\usepackage{pifont} %for the dings
\usepackage{ragged2e}
\usepackage{mathtools}

\usepackage{fontspec}
\setmainfont{Carlito}
\usepackage{tikz}
\usetikzlibrary{positioning}
\usepackage{fontawesome5} % for icons
\setbeamertemplate{navigation symbols}{}


% 
\usepackage{graphicx}
\newcommand\sbullet[1][.5]{\mathbin{\vcenter{\hbox{\scalebox{#1}{$\bullet$}}}}} % for bullets small, tiny, etc.
% https://tex.stackexchange.com/questions/389238/is-there-a-black-dot-symbol-that-i-can-use


%
\title{Entreprise risk management}
%\subtitle{Focus on the bond market}
\author{ \\ 
Foly Ananou, PhD \\
    %\footnotesize{Ph.D. Student, Economics} \\
    %\scriptsize{Université de Montréal and CIREQ} \\
   % \vspace{.25 cm}
   % \normalsize{ \textbf{Supervisor} \\
    %    Immo Schott} \\
   % \footnotesize{Assistant Professor, Economics} \\
   % \scriptsize{Université de Montréal and CIREQ}
    }
%\date{\today}
\date{}
%\titlegraphic{\includegraphics[height=1.5cm, width=\textwidth]{logo.jpg}}

%-------------------------------------------------------
\begin{document}
\begin{frame}[plain] % noframenumbering for beginining at 0
  % Ph.D. Project
    %\vspace{.5 cm}
  %\centering {\tiny \textsc{{\color{myblue} \Ovalbox{\textbf{Oral Exam}}}} }
  \centering
  %\includegraphics[height=1cm, width=0.6\textwidth]{logo.jpg}
    \vspace{1cm}

  \maketitle
\end{frame}

% Uncomment these lines for an automatically generated outline.
%\begin{frame}{Outline}
%  \setbeamertemplate{section in toc}[sections numbered]
%  \tableofcontents%[hideallsubsections]
%\end{frame}

%\AtBeginSection[]
%{
%    \begin{frame}[plain]{Next in the outline}
%        \setbeamertemplate{section in toc}[sections numbered]
%		\tableofcontents[currentsection]%, hidecurrentsubsection, hideothersubsections] %, pausesubsections
%	\end{frame}
%}


\section{Course objectives}

\subsection{Course objectives}
\begin{frame}{Course objectives}
    \begin{itemize}
        \item[$\bullet$] Master ERM fundamentals : understand the principles, frameworks, 
        and benefits of ERM
\medskip

\item[$\bullet$] Analyse and measure risk: learn to identify risk, assess and quantify risks
 using qualitative and quantitative tools 
\medskip

        \item[$\bullet$] Understand the implementation of ERM and best practices
        \medskip


   \end{itemize}
\end{frame}

%*****************************************
\subsection{Chap1}
\begin{frame}{}

\centering
\color{blue} \Ovalbox{\textbf{ERM: fundamentals}}

\end{frame}
%*****************************************
\subsection{Chap1}
\begin{frame}{ Understanding risk: preview}

    \begin{itemize}
        \item [$\bullet$] Let consider the following situations: 
                               \medskip
            \begin{itemize}
                \item[$\bullet$] You are in vacation in Cote d'Ivoire and decide to have an 
                outdoor gathering with friends at the beach. $\rightarrowtail $ What could 
                possibly go wrong ? How likely could this happen ? if it does happen, will it 
                be very harmful ? What can you do to prevent the event from happening or reduce 
                the impact ? 
                       \medskip

 \item[$\bullet$] You got it involve in a car crash $\rightarrowtail $ Is it necessary 
 someone fault or is it just one of those things that happen in life ? 
            \end{itemize}


    \end{itemize}
\end{frame}

%*****************************************
\subsection{Chap1}
\begin{frame}{Understanduing risk: a definition}
    \begin{itemize}
        \item [$\bullet$] Berstein (1996) defines risk as the uncertainty around the 
        outcome of a decision, process or event which can be negative (losses) or 
        positive (benefits or opportunities).

         \centering
  \includegraphics[height=6cm]{risk_bell_curve.png}
    \vspace{1cm} 

    \end{itemize}
\end{frame}

\subsection{Chap1}
\begin{frame}{Understanduing risk: key characteristics of risk}
    \begin{itemize}
        \item [$\bullet$] Uncertainty : outcomes are not guaranteed or predictable
         \medskip 

                 \item [$\bullet$] Dual nature: risk is not only about threats or losses,
                  it can also includes opportunities          \medskip 

                                  \item [$\bullet$] Measurable dimensions: 
    \begin{itemize}
        \item [$\bullet$] \Ovalbox{\textbf{exposure}} $\rightarrowtail $ what is at stake ? 
         \medskip 
          \item [$\bullet$] \Ovalbox{\textbf{probability}} $\rightarrowtail $ how likely is it possible ? 
         \medskip 
          \item [$\bullet$] \Ovalbox{\textbf{severity}} $\rightarrowtail $ how bad (or good) coud it get ? 
         \medskip 
          \item [$\bullet$] \Ovalbox{\textbf{time}} $\rightarrowtail $ how long ? 
         \medskip 
          \item [$\bullet$] \Ovalbox{\textbf{correlation}} $\rightarrowtail $ can it escalate or drive
          other risks ? 
         \medskip 
    \end{itemize}

    \end{itemize}
\end{frame}

\subsection{Chap1}
\begin{frame}{Understanduing risk: categorization}
    \begin{itemize}

                 \item [$\bullet$] Can you identify the risk type in the following sceenarios ? 
                 (let guess together) 
    \begin{itemize}
        \item [$\bullet$] A large international bank experiences a sophisticated ransomware attack
        that encrypts critical customers data and shuts down online banking services for almost 2W. 
         \medskip 
          \item [$\bullet$] The government announces overnight that all cryptoccurrencies transactions
          are banned taking effect immediately, forcing fintech companies to halt operatiions 
          immediately.  
         \medskip 
          \item [$\bullet$] An investment firm relies heavily on a proprietary risk model to allocate 
          capital. During a period of market stress, the model significantly underestimates correlations 
          across asset classes, leading to losses far exceeding management’s stated risk appetite.
         \medskip 
          \item [$\bullet$] An extreme weather event severely damages key production facilities of an
          energy company, forcing a prolonged shutdown. At the same time, regulators announce tighter
          environmental standards, increasing future compliance and investment costs.
    \end{itemize}

    \end{itemize}
\end{frame}

\subsection{Chap1}
\begin{frame}{Understanduing risk: categorization}
    \begin{itemize}

                 \item [$\bullet$] There is no "one-size-fit-all" in terms of risk category :
                 it's dependent on the analysis framework.  
    \begin{itemize}
        \item [$\bullet$] Market risk can be related to financial markets, for firms it could 
        relate to their ability to compete in a given (chosen) market(s)
          \medskip 
          \item [$\bullet$] Business risk can indicate the full scope of risks faced by a firm or just a subset
          (specific) risk related to the type of business the firm is envolved in (insurance risk for example)
         \medskip 
          \item [$\bullet$] Credit risk can include or exclude risk of changes in observed market 
          credit spread -- with some hint also to liquidity risk.  
    \end{itemize}

    \end{itemize}
\end{frame}

\subsection{Chap1}
\begin{frame}{Understanduing risk: financial risk (credit, liquidity and interest-rate)}


    \begin{itemize}

                 \item [$\bullet$] A bank lends money to firms and households, and its funding is essentially based on customers deposits.  
                 \begin{itemize}
        \item [$\bullet$] In what situations does the bank lose money ?
        \begin{itemize}
            \item[$\bullet$]  Total or partial ? Why ? 
            \item[$\bullet$]  Will the losses occur immediately or over time ? 
        \end{itemize} 
          \medskip 
          \item [$\bullet$] In what situation does the bank run out of cash ? 
           \begin{itemize}
            \item[$\bullet$] Does that mean the bank is insolvent ?  
            \item[$\bullet$] Can the bank fail in that situation ? 
        \end{itemize} 
           \medskip 
          \item [$\bullet$] In what situation does the bank profit fluctuate even there is nothing wrong on its customers side ? 
          \begin{itemize}
            \item [$\bullet$] The bank makes 20-year fixed-rate loans at 2\%. It funds itself with deposits whose interest rate can change every year.
What happen if interest rate rise to 5\% ? 
        \end{itemize} 
    \end{itemize}
    \end{itemize}
\end{frame}

\subsection{Chap1}
\subsection{Chap1}
\begin{frame}{Understanduing risk: financial risk (credit, liquidity and interest-rate)}
\begin{columns}[T,onlytextwidth]
\begin{column}{0.52\textwidth}
\begin{itemize}
    \item [$\bullet$] Commonly used and simplest measures are the Value-at-risk (VaR) and Expected Shortfall (ES)
\medskip
\begin{itemize}
  \item[$\bullet$] VaR summarizes potential financial losses within a firm, portfolio,
  or position over a specific time frame. It reflects the worst expected loss
  under a given time horizon \ldots
  \begin{itemize}
    \item[] \ldots given a certain confidence level.
  \end{itemize}
  \medskip
  \item[$\bullet$] Expected shortfall extends VaR by capturing tail risk beyond the VaR cutoff.
   \medskip
\end{itemize}
\end{itemize}
 \href{https://github.com/fsananou/bsb-erm/blob/ebbd7723a3ce60ced169f337c8f0b4d8e8cad564/var_es_exercices.pdf}{\ovalbox{Practice}}

\end{column}

\begin{column}{0.48\textwidth}
\centering
\includegraphics[width=\linewidth]{var.png}
\end{column}
\end{columns}

\end{frame}

\subsection{Chap1}
\subsection{Chap1}
\begin{frame}{Definition of ERM}

    \begin{itemize}

                 \item [$\bullet$] Lam (2003) "ERM is all about integration: ... an integrated risk organisation, ... the integration 
                 of risk transfer strategies, ... the integration of risk management into the business process of a company"
          \medskip 
          \item [$\bullet$] Kemp and Patel (2011) define ERM as a framework, using risk as the core building block, to enable key business decisions 
          to be aligned with inherent riskIt involes hoslistic management of risk and management of business/portfolio as an enterprise. 
           \medskip 
          \item [$\bullet$] Casualty Actuarial Society (2003) "The discipline by which and organization in any industry assesses, controls, 
          exploits, finances and monitors risks from all sources for the purpose of increasing the organization's short and long-term value to its stakeholders."  
          \medskip 
\end{itemize}
             \ovalbox{Discussion: Does ERM create value for a firm ? Why and how ? }

\end{frame}



\subsection{Chap1}
\subsection{Chap1}
\begin{frame}{Evolution of ERM}
    \begin{itemize}

\item[$\bullet$]  ERM has evolved from a control-oriented function
to a strategic framework focused on value creation.

\medskip

\begin{itemize}
    \item[$\bullet$] \textbf{Compliance and risk control} :
    Initial focus on regulatory compliance and adherence to limits, with risks
    managed in silos.

    \item [$\bullet$]\textbf{Loss minimisation} :
    Emphasis on reducing downside outcomes through basic mitigation tools and
    operational controls.

    \item[$\bullet$] \textbf{Risk management} :
    Development of structured risk identification, assessment, and mitigation
    processes across the organization.

    \item[$\bullet$] \textbf{Risk measurement} :
    Introduction of quantitative tools such as economic capital models, stress
    testing, and scenario analysis to measure risk consistently.

    \item [$\bullet$]\textbf{Strategic integration (today)} :
    Integration of risk considerations into strategic decisions, capital
    allocation, and performance management.

    \item[$\bullet$] \textbf{Risk optimisation and value creation} :
    State-of-the-art ERM treats risk as a resource to be actively allocated in
    order to optimise risk--return trade-offs and support long-term value
    creation.
\end{itemize}
\end{itemize}
\end{frame}

\subsection{Chap1}
\subsection{Chap1}
\begin{frame}{Core risk terminology}
\begin{itemize}
    \item[$\bullet$] \textbf{Uncertainty} :
    Lack of complete information about future outcomes, without necessarily
    implying measurable probabilities.

    \item[$\bullet$] \textbf{Risk exposure} :
    The degree to which the organization is subject to a given risk, given its
    activities, assets, and environment.

    \item[$\bullet$] \textbf{Risk appetite} :
    The amount and type of risk an organization is willing to accept in pursuit
    of its strategic objectives.

    \item[$\bullet$] \textbf{Risk tolerance} :
    Acceptable variation around objectives, translating risk appetite into
    operational limits.

    \item[$\bullet$] \textbf{Risk capacity} :
    The maximum level of risk the organization can absorb without threatening
    its viability.

    \item[$\bullet$] \textbf{Residual risk} :
    The level of risk remaining after controls and mitigation actions are applied.

    \item[$\bullet$] \textbf{Inherent risk} :
    The level of risk that exists in the absence of any controls or mitigation.

\end{itemize}

\end{frame}



\begin{frame}{ERM Framework}

\begin{itemize}
    \item[$\bullet$]\textbf{Reference frameworks (COSO, ISO 31000)} :
    Provide a common structure and language for identifying, assessing, and
    governing risks across the organization.
\medskip
    \item[$\bullet$] \textbf{Risk principles and policies} :
    Establish consistent rules for how risks are defined, assessed, reported,
    and escalated.
\medskip

    \item[$\bullet$] \textbf{Risk appetite and tolerance} :
    Translate strategic objectives into explicit limits on acceptable risk-taking
    and risk variability.
\medskip

    \item[$\bullet$] \textbf{Link to strategy and performance} :
    Ensure that risk considerations influence strategic choices, capital
    allocation, and performance evaluation.
\end{itemize}

\end{frame}

\subsection{Chap1}
\subsection{Chap1}
\begin{frame}{ERM Architecture}


\begin{itemize}
    \item [$\bullet$] \textbf{Board of Directors} \\
    Sets risk appetite, oversees risk strategy, and ensures that risk-taking is
    aligned with the organization’s long-term objectives.

    \item [$\bullet$] \textbf{Risk Committee} \\
    Provides focused oversight, challenges management decisions, and ensures
    timely escalation of material risks.

    \item [$\bullet$] \textbf{Executive Management} \\
    Owns risk-taking decisions and integrates risk considerations into day-to-day
    management and strategic planning.

    \item [$\bullet$] \textbf{Chief Risk Officer (CRO)} \\
    Provides an independent, enterprise-wide view of risk and coordinates risk
    management activities across functions.

    \item [$\bullet$] \textbf{Three Lines of Defense} \\
    Clarifies roles between risk-taking (first line), risk oversight and control
    (second line), and independent assurance (third line).
\end{itemize}


\end{frame}


\subsection{Chap1}
\subsection{Chap1}
\begin{frame}{ERM process}
     \includegraphics[width=\linewidth]{erm_process.png}
\vspace{1em}
\ovalbox{Discussion: Is ERM a quantitative or a qualitative tool ?}

\subsection{Chap1}
\subsection{Chap1}
\end{frame} 


\subsection{Chap1}
\subsection{Chap1}
\begin{frame}{ERM: Framework, Architecture and Process (synthesis)}

\begin{description}
    \item[\textbf{Framework}] Defines principles, boundaries, and acceptable
    risk-taking (risk appetite, policies, strategic alignment).

    \item[\textbf{Architecture}] Allocates responsibilities and ensures risk
    oversight (board, CRO, three lines of defense).

    \item[\textbf{Process}] Operationalizes ERM through identification,
    assessment, response, and monitoring of risks.
\end{description}

\begin{block}{Synthesis}
Effective ERM requires consistency between what is allowed
(framework), who is accountable (architecture), and how risks are
managed (process).
\end{block}
\medskip

 \href{https://github.com/fsananou/bsb-erm/blob/main/case_studies.pdf}{\ovalbox{Practice}}

\end{frame}

\subsection{Chap1}
\subsection{Chap1}
\begin{frame}{Why ERM Fails in Practice (synthesis)}
\begin{itemize}
 \item[$\bullet$] ERM failures typically reflect governance and judgment issues rather than
technical deficiencies.

\medskip

\begin{itemize}
    \item[$\bullet$] \textbf{ERM reduced to compliance} \\
    Frameworks exist formally but do not influence strategic decisions.

    \item[$\bullet$] \textbf{Unclear or ineffective risk appetite} \\
    Limits are vague, ignored, or overridden during growth phases.

    \item[$\bullet$] \textbf{Misaligned incentives} \\
    Short-term performance is rewarded while risk accumulation is not penalized.

    \item[$\bullet$] \textbf{Weak challenge and escalation} \\
    Risk signals are diluted, delayed, or dismissed when inconvenient.

    \item[$\bullet$] \textbf{Illusion of control through models} \\
    Quantitative tools create false confidence and mask tail risks.
\end{itemize}
\end{itemize}

\begin{block}{Key takeaway}
ERM fails when risk governance is disconnected from decision-making,
incentives, and organizational culture.
\end{block}

\end{frame}

%*****************************************
\subsection{Chap1}
\begin{frame}{}

\centering
\color{blue} \Ovalbox{\textbf{ERM frameworks and architecture}}

\end{frame}
%*****************************************

\begin{frame}{ERM Frameworks: What problem do they Solve?}

Organizations face many risks simultaneously, across units, time horizons,
and objectives.

\medskip

ERM frameworks exist to solve three coordination problems:

\begin{itemize}
    \item \textbf{Cognitive problem:} Different actors perceive and describe
    risk differently.
    
    \item \textbf{Organizational problem:} Risks are generated locally but
    consequences are enterprise-wide.
    
    \item \textbf{Governance problem:} Risk-taking decisions are often separated
    from risk-bearing consequences.
\end{itemize}

\begin{block}{Core function}
An ERM framework aligns perception, responsibility, and decision-making
around risk.
\end{block}

\end{frame}
\begin{frame}{COSO ERM: Why the control orientation?}

COSO ERM emerged from repeated failures in financial reporting and governance.

\medskip

\begin{itemize}
    \item The primary concern is \textbf{accountability}: who is responsible
    when objectives are missed?
    
    \item Risk is framed as a \textbf{source of deviation} from expected outcomes.
    
    \item Emphasis is placed on documentation, traceability, and auditability.
    
    \item The framework assumes that better controls reduce unacceptable risk.
\end{itemize}

\begin{block}{Implicit assumption}
Risk failures are largely due to weak controls or poor oversight.
\end{block}

\end{frame}
\begin{frame}{COSO ERM: Where the logic breaks}

The control-based logic has structural limits.

\medskip

\begin{itemize}
    \item Not all risks are controllable ex ante (strategic, systemic, tail risks).
    
    \item Excessive controls can delay decisions and suppress risk signals.
    
    \item Documentation may substitute for genuine challenge.
    
    \item Strategic risk-taking can be discouraged even when value-creating.
\end{itemize}

\begin{block}{Key insight}
Strong controls do not guarantee good risk decisions.
\end{block}

\end{frame}

\begin{frame}{ISO 31000: Risk as a Decision Problem}

ISO 31000 starts from a fundamentally different premise.

\medskip

\begin{itemize}
    \item Risk is defined as the \textbf{effect of uncertainty on objectives},
    not merely as potential loss.
    
    \item Risk management is meaningful only if it influences decisions.
    
    \item There is no universally optimal risk process — context matters.
    
    \item Judgment is unavoidable and must be structured, not eliminated.
\end{itemize}

\begin{block}{Implicit assumption}
Risk cannot be fully controlled, only understood and governed.
\end{block}

\end{frame}

\begin{frame}{ISO 31000: Why flexibility becomes a weakness}

Without strong governance, ISO-based ERM can drift.

\medskip

\begin{itemize}
    \item Risk appetite remains qualitative and non-binding.
    
    \item Different units interpret principles inconsistently.
    
    \item Risk discussions lack escalation and enforcement.
    
    \item Strategic narratives replace quantitative discipline.
\end{itemize}

\begin{block}{Failure mode}
Risk is discussed but not constrained.
\end{block}

\end{frame}

%========================
% Slide 1: COSO vs ISO 31000 (deep comparison)
%========================
\begin{frame}{COSO ERM vs ISO 31000: Two philosophies of ERM}

\begin{columns}[T,onlytextwidth]

%--- COSO column
\begin{column}{0.5\textwidth}
    \begin{itemize}
  \item[$\bullet$]  \textbf{COSO ERM}

\begin{itemize}
  \item[$\bullet$] \textbf{Primary logic: accountability and control} --
  Designed to reduce deviations from objectives through clear governance,
  internal control, and traceability.

  \item[$\bullet$] \textbf{Strength: auditability and discipline} --
  Produces roles, documentation, and evidentiary trails that work well in
  regulated or listed environments.

  \item[$\bullet$] \textbf{Typical failure mode: compliance substitution} --
  When treated as a checklist, the firm may appear ``in control'' while
  strategic and tail risks remain unchallenged.
\end{itemize}
    \end{itemize}

\end{column}

%--- ISO column
\begin{column}{0.5\textwidth}
        \begin{itemize}
\item \textbf{ISO 31000}

\begin{itemize}
  \item[$\bullet$] \textbf{Primary logic: decisions under uncertainty} --
  Risk is the effect of uncertainty on objectives; ERM help shape choices, trade-offs, and priorities.

  \item[$\bullet$] \textbf{Strength: integration and flexibility} --
  Principles-based design encourages embedding risk into strategy, planning,
  and day-to-day decisions across diverse contexts.

  \item[$\bullet$] \textbf{Typical failure mode: dilution and inconsistency} --
  Without strong governance, principles can become vague, unevenly applied,
  and non-binding for risk-taking.
\end{itemize}
    \end{itemize}
\end{column}

\end{columns}

\begin{block}{Takeaway}
COSO $\rightarrowtail $ \emph{discipline, controls, and assurance} are
the priority; ISO $\rightarrowtail $ \emph{decision integration and strategic
adaptation} are the priority. 
\end{block}

\end{frame}



\begin{frame}{Risk apetite: How it works in practice}

Risk appetite is a \textbf{governance mechanism}, not a slogan.

\medskip

\begin{itemize}
    \item It translates strategic ambition into acceptable uncertainty.
    
    \item It constrains risk-taking before losses materialize.
    
    \item It provides a reference for escalation and challenge.
    
    \item It must be expressed in both qualitative and quantitative terms.
\end{itemize}

\begin{block}{Key mechanism}
Risk appetite defines when risk-taking becomes a governance issue.
\end{block}

\end{frame}

\begin{frame}{Confusing Appetite, Tolerance and Capacity is dangerous}

These concepts play different roles in ERM.

\medskip

\begin{itemize}
    \item Risk appetite reflects \textbf{strategic choice}.
    
    \item Risk tolerance reflects \textbf{operational control}.
    
    \item Risk capacity reflects \textbf{survival constraints}.
\end{itemize}

\medskip

\begin{itemize}
    \item Exceeding tolerance requires management action.
    \item Exceeding appetite signals strategic drift.
    \item Exceeding capacity threatens viability.
\end{itemize}

\begin{block}{Crisis pattern}
Most failures occur when appetite silently converges toward capacity.
\end{block}

\end{frame}

\begin{frame}{ERM Maturity models: The hidden trap}

ERM maturity models suggest linear progress.

\medskip

\begin{itemize}
    \item They implicitly assume that more integration is always better.
    
    \item They reward formalization over effectiveness.
    
    \item They can encourage cosmetic improvements.
    
    \item They underestimate the role of context and strategy.
\end{itemize}

\begin{block}{Key warning}
High ERM maturity does not immunize against failure.
\end{block}

\end{frame}
\begin{frame}{Why ERM frameworks become box-ticking}

Box-ticking is a rational organizational outcome.

\medskip

\begin{itemize}
    \item Framework adoption is often driven by regulation or reputation.
    
    \item Incentives favor formal compliance over substantive challenge.
    
    \item Responsibility for risk is diffused.
    
    \item Negative information is costly to escalate.
\end{itemize}

\begin{block}{Structural insight}
Box-ticking is not a mistake — it is a governance failure.
\end{block}

\end{frame}
%========================
% Slide 2: What makes an effective ERM framework?
%========================
\begin{frame}{What Makes an Effective ERM Framework?}

\begin{itemize}
\item An effective ERM framework is not the presence of documents; it is a system
that \textbf{changes decisions} and \textbf{constrains risk-taking} before losses occur.

\begin{itemize}
  \item[$\bullet$] \textbf{Clear objectives and risk taxonomy} --
  Risks are defined relative to objectives, using a common language that avoids
  silo interpretations and enables aggregation at enterprise level.

  \item[$\bullet$] \textbf{Explicit risk appetite linked to strategy} --
  Appetite expresses the acceptable uncertainty required to pursue strategy;
  it becomes meaningful when translated into operational limits and triggers.

  \item[$\bullet$] \textbf{Decision-usefulness (not reporting volume)} --
  Risk information must answer: \emph{What changes in our decision today?}
  Otherwise ERM becomes descriptive rather than prescriptive.

  \item[$\bullet$] \textbf{Escalation and consequence mechanisms} --
  Effective frameworks define when issues must be escalated and what actions
  follow (reduce exposure, revise limits, halt activity, strengthen controls).

  \item[$\bullet$] \textbf{Balance of quantitative and judgment} --
  Models structure discipline; judgment handles regime shifts, tail risks, and
  what data cannot capture.
\end{itemize}
\end{itemize}

\begin{block}{Litmus test}
If risk appetite breaches do not trigger action and debate at the right level,
the framework is cosmetic.
\end{block}

\end{frame}
%========================
% Slide 3: Transition to Architecture
%========================
\begin{frame}{From Framework to Architecture}

A framework specifies \textbf{what should happen}.  
Architecture determines \textbf{who makes it happen} and \textbf{how power and information flow}.

\medskip

\begin{itemize}
  \item[$\bullet$] \textbf{Who owns risk? (risk-taking authority)}\\
  Business units typically originate risk; architecture clarifies the limits of
  their autonomy and when decisions become governance issues.

  \item[$\bullet$] \textbf{Who challenges risk? (independent oversight)}\\
  The CRO and second-line functions must have independence, access, and standing
  to challenge decisions that conflict with appetite or capacity.

  \item[$\bullet$] \textbf{Who assures risk controls? (independent assurance)}\\
  Internal audit validates that controls and reporting are effective, not only
  present, and that exceptions are properly managed.

  \item[$\bullet$] \textbf{How escalation works under pressure}\\
  Architecture defines escalation paths that still function when incentives
  push toward ignoring bad news.
\end{itemize}

\end{frame}
\begin{frame}{ERM Architecture: Core functions}

ERM architecture is a governance system that performs four critical functions.

\medskip

\begin{itemize}
  \item[$\bullet$] \textbf{Allocation of decision rights}\\
  Determines who can take risk, within which limits, and under what conditions.

  \item[$\bullet$] \textbf{Independent challenge}\\
  Ensures that risk-taking decisions are reviewed and contested by actors
  without direct profit incentives.

  \item[$\bullet$] \textbf{Information and escalation}\\
  Guarantees that material risks reach the appropriate decision level in time.

  \item[$\bullet$] \textbf{Assurance and accountability}\\
  Verifies that controls work and assigns responsibility when outcomes diverge.
\end{itemize}

\begin{block}{Organizing principle}
Architecture is about enforcement, not reporting.
\end{block}

\end{frame}

\begin{frame}{Decision rights and Risk-taking}

Risk originates from decisions, not models.

\medskip

\begin{itemize}
  \item[$\bullet$] Business units make local decisions that generate enterprise-wide
  risk exposure.

  \item[$\bullet$] Risk-taking authority is delegated but constrained by mandates,
  limits, and escalation triggers.

  \item[$\bullet$] When decision rights are implicit or informal, risk accumulates
  outside governance boundaries.
\end{itemize}

\begin{block}{Failure pattern}
Most risk is taken where governance is weakest.
\end{block}

\end{frame}

\begin{frame}{Independent challenge and oversight}

Effective ERM requires structured disagreement.

\medskip

\begin{itemize}
  \item[$\bullet$] The CRO provides an enterprise-wide perspective and challenges
  business decisions that conflict with appetite.

  \item[$\bullet$] Independence is necessary but insufficient; influence and access
  to decision-makers are equally critical.

  \item[$\bullet$] Challenge fails when it is perceived as advisory rather than
  consequential.
\end{itemize}

\begin{block}{Key tension}
The CRO has responsibility without direct authority.
\end{block}

\end{frame}

\begin{frame}{Information flow and escalation}

ERM architecture lives or dies through escalation.

\medskip

\begin{itemize}
  \item[$\bullet$] Risk information must travel upward without filtering,
  delay, or strategic framing.

  \item[$\bullet$] Escalation thresholds define when risk becomes a governance
  issue rather than an operational one.

  \item[$\bullet$] Escalation mechanisms must still function when incentives
  push toward silence.
\end{itemize}

\begin{block}{Diagnostic test}
Would bad news still be escalated if it threatened careers?
\end{block}

\end{frame}

\begin{frame}{Assurance and accountability}

Controls without assurance create an illusion of safety.

\medskip

\begin{itemize}
  \item[$\bullet$] Internal audit independently verifies the effectiveness of
  risk controls and reporting.

  \item[$\bullet$] Assurance focuses on whether controls work in practice,
  not whether they exist on paper.

  \item[$\bullet$] Clear accountability strengthens incentives to respect limits
  and escalation rules.
\end{itemize}

\begin{block}{Common mistake}
Auditing processes instead of risk outcomes.
\end{block}

\end{frame}

\begin{frame}{ERM Architecture under pressure}

ERM architecture is revealed when conditions deteriorate.

\medskip

\begin{itemize}
  \item[$\bullet$] Growth phases weaken constraints and encourage overrides.
  \item[$\bullet$] Stress exposes information bottlenecks and authority gaps.
  \item[$\bullet$] Weak architectures fail silently before losses appear.
\end{itemize}

\begin{block}{Final takeaway}
Architecture determines whether ERM constrains behavior or merely documents it.
\end{block}

\end{frame}

\begin{frame}{ERM architecture in banks: why it is specific}

Banking ERM architecture is shaped by the interaction between
risk-taking, capital, and financial stability.

\medskip

\begin{itemize}
  \item[$\bullet$] Banks operate with \textbf{high leverage}, making small
  shocks potentially systemic rather than firm-specific.

  \item[$\bullet$] Risk-taking decisions directly affect \textbf{capital adequacy},
  liquidity, and solvency.

  \item[$\bullet$] ERM architecture must therefore satisfy not only shareholders,
  but also \textbf{supervisors and the financial system}.
\end{itemize}

\begin{block}{Key implication}
In banks, ERM architecture is inseparable from regulation and supervision.
\end{block}

\end{frame}

\begin{frame}{Basel framework as an architectural constraint}

The Basel framework embeds risk constraints into bank governance.

\medskip

\begin{itemize}
  \item[$\bullet$] Capital requirements limit the amount of risk that can be
  absorbed by the balance sheet.

  \item[$\bullet$] Risk-weighted assets translate heterogeneous risks into
  a common capital metric.

  \item[$\bullet$] Liquidity and leverage ratios constrain funding structures
  and balance sheet expansion.

  \item[$\bullet$] These constraints shape risk-taking incentives before
  management discretion intervenes.
\end{itemize}

\begin{block}{Architectural role}
Basel rules act as ex ante enforcement of risk appetite at system level.
\end{block}

\end{frame}

\begin{frame}{ICAAP: linking ERM to capital and stress}

The Internal Capital Adequacy Assessment Process (ICAAP)
is the core internal ERM mechanism in banks.

\medskip

\begin{itemize}
  \item[$\bullet$] ICAAP translates the bank’s risk profile into internal
  capital needs under normal and stressed conditions.

  \item[$\bullet$] It forces management to articulate assumptions about
  risk, correlations, and tail events.

  \item[$\bullet$] Stress testing within ICAAP reveals vulnerabilities
  not captured by standard risk metrics.

  \item[$\bullet$] ICAAP connects risk appetite, capital planning, and strategy.
\end{itemize}

\begin{block}{Failure pattern}
ICAAP fails when it becomes a regulatory document rather than a
decision-making tool.
\end{block}

\end{frame}

\begin{frame}{Resolution and SRB: architecture under failure}

ERM architecture in banks extends beyond survival scenarios.

\medskip

\begin{itemize}
  \item[$\bullet$] Resolution frameworks assume that banks can fail and
  design mechanisms to contain systemic impact.

  \item[$\bullet$] The Single Resolution Board (SRB) evaluates resolvability
  and loss-absorbing capacity.

  \item[$\bullet$] ERM must therefore consider not only risk prevention,
  but also \textbf{orderly failure}.

  \item[$\bullet$] Risk-taking inconsistent with resolvability constraints
  undermines the credibility of ERM.
\end{itemize}

\begin{block}{Architectural insight}
In banking, ERM must be credible even in failure.
\end{block}

\end{frame}

\begin{frame}{Synthesis: banking ERM as layered enforcement}

\begin{itemize}
  \item[$\bullet$] Basel sets system-wide minimum constraints.
  \item[$\bullet$] ICAAP internalizes risk into capital planning.
  \item[$\bullet$] Governance bodies arbitrate trade-offs.
  \item[$\bullet$] Resolution frameworks discipline excessive risk-taking.
\end{itemize}

\begin{block}{Key takeaway}
Banking ERM architecture is effective only when regulatory,
internal, and governance layers reinforce each other.
\end{block}

\end{frame}
\begin{frame}{Risk culture: the missing enforcement layer}

Formal architecture cannot fully constrain behavior.

\medskip

\begin{itemize}
  \item[$\bullet$] Risk culture shapes how individuals interpret rules,
  limits, and escalation thresholds.

  \item[$\bullet$] It determines whether risk signals are amplified or
  suppressed within the organization.

  \item[$\bullet$] Culture becomes decisive when rules are ambiguous
  or incomplete.
\end{itemize}

\begin{block}{Lam’s insight}
Risk culture determines how ERM functions when no one is watching.
\end{block}

\end{frame}

\begin{frame}{Incentives and risk-taking behavior}

Incentives translate abstract objectives into concrete behavior.

\medskip

\begin{itemize}
  \item[$\bullet$] Compensation structures influence risk-taking more
  powerfully than formal policies.

  \item[$\bullet$] Short-term performance incentives encourage risk
  accumulation that may only materialize later.

  \item[$\bullet$] Poorly designed incentives weaken escalation and
  encourage limit overrides.
\end{itemize}

\begin{block}{Behavioral mechanism}
People manage what they are rewarded for.
\end{block}

\end{frame}
\begin{frame}{Tone from the top and challenge culture}

Leadership behavior sets the effective risk boundary.

\medskip

\begin{itemize}
  \item[$\bullet$] Tone from the top signals whether risk management
  is valued or tolerated.

  \item[$\bullet$] A strong challenge culture allows disagreement
  without career penalties.

  \item[$\bullet$] Silence in meetings often reflects incentives,
  not consensus.
\end{itemize}

\begin{block}{Diagnostic question}
Are people rewarded for raising inconvenient risk concerns?
\end{block}

\end{frame}

\begin{frame}{Why ERM still fails despite strong architecture}

Even robust ERM architectures can collapse.

\medskip

\begin{itemize}
  \item[$\bullet$] Incentives overpower formal constraints.
  \item[$\bullet$] Escalation is perceived as costly or futile.
  \item[$\bullet$] Success reinforces overconfidence and risk drift.
\end{itemize}

\begin{block}{Final takeaway}
ERM fails when incentives and culture neutralize governance.
\end{block}
\medskip 

 \href{https://github.com/fsananou/bsb-erm/blob/main/case_studies.pdf}{\ovalbox{Practice}}
\end{frame}

\subsection{Chap1}
\begin{frame}{}

\centering
\color{blue} \Ovalbox{\textbf{ERM and decision-making}}

\end{frame}

\begin{frame}{Why risk-adjusted performance metrics exist}

Accounting performance measures ignore the uncertainty underlying returns.

\medskip

\begin{itemize}
  \item[$\bullet$] Two activities can generate identical profits while exposing
  the organization to very different downside risks.

  \item[$\bullet$] Risk-adjusted performance metrics explicitly relate expected
  returns to the amount of risk or capital required to sustain them.

  \item[$\bullet$] Their primary function is to make heterogeneous activities
  comparable from a decision-maker’s perspective.
\end{itemize}

\begin{block}{Decision purpose}
Risk-adjusted metrics enable ranking and allocation, not prediction.
\end{block}

\end{frame}

\begin{frame}{RAROC and EVA: what they measure and why they exist}

RAROC and EVA are economic performance metrics designed to link
risk, capital, and value creation.

\medskip

\begin{itemize}
  \item[$\bullet$] \textbf{RAROC (Risk-Adjusted Return on Capital)} measures
  the expected return of an activity relative to the amount of economic capital
  required to absorb unexpected losses. It answers the question:
  \emph{How much return do we earn per unit of risk-bearing capacity consumed?}

  \item[$\bullet$] \textbf{EVA (Economic Value Added)} measures value creation
  by comparing operating profits to the full cost of capital, including the
  opportunity cost of bearing risk. It answers the question:
  \emph{Does this activity create value once risk is fully priced?}

  \item[$\bullet$] Used together, RAROC supports \textbf{relative ranking and
  capital allocation}, while EVA supports \textbf{value creation assessment}
  at firm or business-unit level.
\end{itemize}

\begin{block}{Important clarification}
RAROC and EVA structure decision-making under uncertainty; they are not
precise valuation tools.
\end{block}


\end{frame}

\begin{frame}{ERM as a strategic discipline}

Strategy is inseparable from risk-taking.

\medskip

\begin{itemize}
  \item[$\bullet$] Strategic choices determine which risks the organization
  deliberately accepts in pursuit of value.

  \item[$\bullet$] ERM makes these choices explicit, coherent, and consistent
  with risk-bearing capacity.

  \item[$\bullet$] When conditions change, ERM provides the basis for revising
  strategy rather than reacting after losses materialize.
\end{itemize}

\begin{block}{Final insight}
ERM is effective when it shapes strategy ex ante, not when it explains
failures ex post.
\end{block}

\end{frame}

\begin{frame}{Why risk-adjusted decision-making still fails}

Even when risk-adjusted metrics are available, decisions may ignore them.

\medskip

\begin{itemize}
  \item[$\bullet$] Incentives may favor short-term accounting performance over
  risk-adjusted value creation.

  \item[$\bullet$] Portfolio effects and correlations may be underestimated,
  leading to excessive concentration.

  \item[$\bullet$] Model assumptions may understate tail risks, especially
  during benign periods.

  \item[$\bullet$] Governance may allow exceptions that neutralize discipline.
\end{itemize}

\begin{block}{Link to ERM}
Risk-adjusted metrics require architecture, incentives, and culture
to be effective.
\end{block}

\end{frame}

\begin{frame}{Exercise 1: risk-adjusted performance and allocation}

A bank considers three business activities competing for limited capital.
Expected annual figures are shown below.

\medskip

\centering
\begin{tabular}{lccc}
\hline
 & Activity A & Activity B & Activity C \\
\hline
Expected profit (€m) & 12 & 9 & 6 \\
Economic capital (€m) & 60 & 30 & 15 \\
\hline
\end{tabular}

\medskip

The bank has a total economic capital budget of €75m
and a target RAROC of 15\%.

\begin{enumerate}
  \item Compute the RAROC of each activity.

  \item Rank the activities based on their risk-adjusted performance.

  \item Given the capital constraint, which activities should the bank
  prioritize?

  \item Is it optimal to select only the activity with the highest RAROC?
  Explain.

  \item How does this exercise illustrate the role of ERM in capital allocation?
\end{enumerate}

\end{frame}


\begin{frame}{Exercise 2: portfolio view and strategy}

The bank currently operates Activity B and Activity C from Exercise 1.
Management considers adding Activity A to boost profitability.

\medskip

Additional information:
\begin{itemize}
  \item[$\bullet$] Activity A and B are exposed to the same economic sector.
  \item[$\bullet$] Activity C is weakly correlated with A and B.
  \item[$\bullet$] Under adverse conditions, losses on A and B tend to
  materialize simultaneously.
\end{itemize}

The bank’s stated strategy emphasizes resilience and stable earnings.


\begin{enumerate}
  \item From a standalone perspective, does Activity A appear attractive?
  Why?

  \item How do correlation and concentration change the risk assessment at
  portfolio level?

  \item Why might a portfolio-level perspective contradict the RAROC ranking?

  \item Is the proposed expansion consistent with the bank’s stated strategy?

  \item What additional information would ERM require before approving
  this decision?
\end{enumerate}

\end{frame}

\subsection{Chap1}
\begin{frame}{}

\centering
\color{blue} \Ovalbox{\textbf{ERM and core risk categories}}

\end{frame}
\begin{frame}{Credit risk in ERM}

Credit risk is the risk of loss arising from counterparty failure.

\medskip

\begin{itemize}
  \item[$\bullet$] Traditional credit risk focuses on probability of default
  and loss given default at individual exposure level.

  \item[$\bullet$] From an ERM perspective, concentration, correlation, and
  cyclicality are often more dangerous than isolated defaults.

  \item[$\bullet$] Credit risk interacts strongly with market conditions,
  liquidity, and strategy during downturns.
\end{itemize}

\begin{block}{ERM role}
Ensure credit growth, portfolio concentration, and capital planning remain
consistent with risk appetite.
\end{block}

\end{frame}

\begin{frame}{How ERM governs credit risk hedging}

ERM governs credit risk hedging as a portfolio and capital decision.

\medskip

\begin{itemize}
  \item[$\bullet$] \textbf{Risk appetite and scope:} ERM determines which credit
  risks are core to the business model and should be retained, and which
  exposures (e.g.\ sectoral or single-name concentration) should be reduced.

  \item[$\bullet$] \textbf{Hedging instruments:} ERM guides the use of
  collateral, guarantees, credit insurance, credit default swaps (CDS),
  securitization, or portfolio sales depending on objectives.

  \item[$\bullet$] \textbf{Portfolio focus:} Hedging targets correlation,
  concentration, and tail exposure rather than individual defaults.

  \item[$\bullet$] \textbf{Secondary risks:} ERM assesses counterparty risk,
  liquidity risk, basis risk, and legal risk introduced by credit hedges.

  \item[$\bullet$] \textbf{Governance:} Limits are set on hedge reliance,
  counterparty exposure, and complex structures, with escalation for exceptions.
\end{itemize}

\begin{block}{ERM principle}
Credit hedging arbitrates risk transfer versus capital usage at enterprise level.
\end{block}

\end{frame}

\begin{frame}{Market risk in ERM}

Market risk arises from changes in prices, rates, and spreads.

\medskip

\begin{itemize}
  \item[$\bullet$] Day-to-day volatility is usually manageable and expected.

  \item[$\bullet$] The main ERM challenge lies in tail events, correlation
  breakdowns, and forced deleveraging.

  \item[$\bullet$] Market risk becomes systemic when combined with leverage
  and liquidity constraints.
\end{itemize}

\begin{block}{ERM role}
Prevent market risk from amplifying into solvency or liquidity crises.
\end{block}

\end{frame}

\begin{frame}{How ERM governs market risk hedging}

Market risk hedging balances volatility reduction with liquidity and strategy.

\medskip

\begin{itemize}
  \item[$\bullet$] \textbf{Risk tolerance:} ERM defines acceptable earnings
  volatility and tail losses, which determines the extent of hedging.

  \item[$\bullet$] \textbf{Hedging instruments:} Forwards, futures, swaps,
  options, and structured derivatives are used to reshape exposure profiles.

  \item[$\bullet$] \textbf{Hedging design:} ERM distinguishes between linear
  hedges (cost-efficient) and optional hedges (tail protection).

  \item[$\bullet$] \textbf{Secondary risks:} ERM evaluates margin calls,
  liquidity needs, basis risk, and counterparty exposure arising from hedges.

  \item[$\bullet$] \textbf{Governance:} ERM limits speculative hedging,
  complexity, and model-dependent strategies.
\end{itemize}

\begin{block}{ERM principle}
Market hedging reduces volatility but must preserve liquidity and resilience.
\end{block}

\end{frame}


\begin{frame}{Operational risk in ERM}

Operational risk arises from failures in processes, systems, people, or governance.

\medskip

\begin{itemize}
  \item[$\bullet$] Operational risk is often treated as residual, but it
  frequently triggers or amplifies other risk categories.

  \item[$\bullet$] Weak controls, poor data, or governance failures can turn
  manageable risks into severe losses.

  \item[$\bullet$] Culture, incentives, and decision processes are key drivers
  of operational risk.
\end{itemize}

\begin{block}{ERM role}
Identify structural weaknesses that cut across all risk categories.
\end{block}

\end{frame}

\begin{frame}{How ERM governs operational risk mitigation}

Operational risk is mitigated primarily through governance and controls.

\medskip

\begin{itemize}
  \item[$\bullet$] \textbf{Risk identification:} ERM identifies critical
  processes whose failure could amplify financial or strategic risks.

  \item[$\bullet$] \textbf{Instruments and tools:} Insurance contracts,
  outsourcing agreements, redundancy systems, and business continuity plans.

  \item[$\bullet$] \textbf{Control design:} ERM prioritizes investments in
  systems, controls, and data quality based on risk materiality.

  \item[$\bullet$] \textbf{Secondary risks:} ERM evaluates dependency risk,
  legal risk, and residual exposure not covered by insurance.

  \item[$\bullet$] \textbf{Governance:} Clear accountability and reporting
  lines are enforced for operational incidents.
\end{itemize}

\begin{block}{ERM principle}
Operational risk mitigation improves how the organization functions.
\end{block}

\end{frame}

\begin{frame}{Liquidity risk in ERM}

Liquidity risk arises when obligations cannot be met without severe cost.

\medskip

\begin{itemize}
  \item[$\bullet$] Liquidity risk materializes abruptly and is highly
  non-linear.

  \item[$\bullet$] It often results from interactions between market stress,
  funding structures, and confidence.

  \item[$\bullet$] Liquidity crises frequently occur despite compliance with
  other risk limits.
\end{itemize}

\begin{block}{ERM role}
Ensure that growth, leverage, and funding strategies remain resilient
under stress.
\end{block}

\end{frame}

\begin{frame}{How ERM governs liquidity risk mitigation}

Liquidity risk is governed structurally rather than hedged transaction by transaction.

\medskip

\begin{itemize}
  \item[$\bullet$] \textbf{Risk tolerance:} ERM defines survival horizons and
  stress scenarios under which liquidity must be preserved.

  \item[$\bullet$] \textbf{Instruments and tools:} High-quality liquid assets
  (HQLA), committed credit lines, central bank facilities, and funding
  diversification.

  \item[$\bullet$] \textbf{Structural constraints:} ERM constrains maturity
  transformation, leverage, and reliance on short-term funding.

  \item[$\bullet$] \textbf{Secondary risks:} ERM assesses rollover risk,
  collateral availability, and market access under stress.

  \item[$\bullet$] \textbf{Governance:} Contingency funding plans and escalation
  protocols are embedded in ERM.
\end{itemize}

\begin{block}{ERM principle}
Liquidity resilience is achieved through structure, not derivatives.
\end{block}

\end{frame}


\begin{frame}{Strategic risk in ERM}

Strategic risk arises from flawed assumptions, choices, or execution.

\medskip

\begin{itemize}
  \item[$\bullet$] Strategic risk is forward-looking and often difficult
  to quantify.

  \item[$\bullet$] It reflects misalignment between strategy, environment,
  and risk-bearing capacity.

  \item[$\bullet$] Most major failures are strategic in nature, even if
  losses appear as credit or market losses.
\end{itemize}

\begin{block}{ERM role}
Challenge strategic assumptions and test resilience under alternative scenarios.
\end{block}

\end{frame}

\begin{frame}{How ERM governs strategic risk}

Strategic risk is governed through flexibility and disciplined choice.

\medskip

\begin{itemize}
  \item[$\bullet$] \textbf{Risk framing:} ERM challenges strategic assumptions
  using scenarios and stress testing rather than point forecasts.

  \item[$\bullet$] \textbf{Instruments and mechanisms:} Real options, staged
  investments, exit clauses, diversification, partnerships, and joint ventures.

  \item[$\bullet$] \textbf{Irreversibility control:} ERM limits commitments that
  could lock the organization into fragile strategies.

  \item[$\bullet$] \textbf{Secondary risks:} ERM evaluates reputational,
  political, and long-horizon risks created by strategic choices.

  \item[$\bullet$] \textbf{Governance:} Strategic decisions are subjected to
  explicit risk challenge at board level.
\end{itemize}

\begin{block}{ERM principle}
Strategic risk is reduced by preserving optionality, not by eliminating risk.
\end{block}

\end{frame}


\begin{frame}{Risk categories are not silos}

Enterprise risk emerges from interactions.

\medskip

\begin{itemize}
  \item[$\bullet$] Credit stress can trigger market losses and liquidity runs.

  \item[$\bullet$] Operational failures can magnify all other risks.

  \item[$\bullet$] Strategic errors often determine the scale of losses.
\end{itemize}

\begin{block}{Key message}
ERM exists to arbitrate interactions, not to optimize silos.
\end{block}

\end{frame}


\begin{frame}{Derivatives in ERM: why they are ubiquitous}

Derivatives are the dominant instruments for hedging because they allow
precise and flexible risk transfer.

\medskip

\begin{itemize}
  \item[$\bullet$] Derivatives isolate specific risk factors (rates, prices,
  spreads, FX) without requiring changes to the underlying business activity.

  \item[$\bullet$] They enable rapid adjustment of exposures, which is
  particularly valuable when risks evolve faster than balance sheets.

  \item[$\bullet$] Compared to structural changes, derivatives are often
  capital-efficient and operationally convenient.
\end{itemize}

\begin{itemize}
  \item[$\bullet$] However, derivatives do not remove risk from the system;
  they transform it into counterparty, liquidity, model, and governance risks.

  \item[$\bullet$] As a result, derivative usage is not a technical decision
  but an enterprise risk decision.
\end{itemize}

\begin{block}{ERM perspective}
Derivatives are powerful risk-shaping tools, but they require strict governance
to avoid amplifying enterprise risk.
\end{block}

\end{frame}

\begin{frame}{Hedging assessment}

ERM assesses hedging both \emph{before} and \emph{after} implementation.

\medskip

\begin{itemize}
  \item[$\bullet$] \textbf{Unhedged and hedged positions:}
  Let $X$ denote the unhedged exposure and $H$ the hedging instrument.
  The hedged position is defined as:
  \[
  Y = X - hH
  \]
  where $h$ is the hedge ratio chosen by the firm.

  \item[$\bullet$] \textbf{Hedge ratio (definition):}
  The hedge ratio represents the quantity of hedging instrument used per unit
  of exposure. A common benchmark is the minimum-variance hedge ratio:
  \[
  h^* = \frac{\mathrm{Cov}(X,H)}{\mathrm{Var}(H)}
  \]
  used as a reference rather than a mechanical rule in ERM.
\end{itemize}

  \end{frame}

\begin{frame}{Hedging assessment}
\begin{itemize}

  \item[$\bullet$] \textbf{Prospective effectiveness (ex ante):}
  ERM evaluates whether the hedge is expected to reduce risk:
  \[
  \text{HE}_{\text{pros}} = 1 - 
  \frac{\mathrm{Var}(Y)}{\mathrm{Var}(X)}
  \]
  based on assumptions about correlation, volatility, and stress scenarios.

  \item[$\bullet$] \textbf{Retrospective effectiveness (ex post):}
  After implementation, ERM reassesses realized performance:
  \[
  \text{HE}_{\text{retro}} = 1 -
  \frac{\mathrm{Var}(X - hH)}{\mathrm{Var}(X)}
  \]
  to detect model drift, breakdowns, or unintended risk amplification.

  \end{itemize}

  \end{frame}

\begin{frame}{Hedging assessment}
\begin{itemize}

  \item[$\bullet$] \textbf{ERM extension to tail risk:}
  Effectiveness is also assessed on downside risk:
  \[
  \Delta ES = ES(X) - ES(Y)
  \]
  complemented by analysis of liquidity and margin impacts.
\end{itemize}

\begin{block}{ERM interpretation}
A hedge is acceptable only if prospective and retrospective effectiveness
remain robust under stress and do not create secondary enterprise risks.
\end{block}

\medskip 
\medskip 

 \href{https://github.com/fsananou/bsb-erm/blob/main/case_studies.pdf}{\ovalbox{Practice}}

\end{frame}

\begin{frame}{}

\centering
\color{blue} \Ovalbox{\textbf{ERM in practice}}

\end{frame}
\begin{frame}{Why ERM implementation is structurally difficult}

ERM implementation fails not because principles are unclear, but because
it conflicts with how organizations actually operate.

\medskip

\begin{itemize}
  \item[$\bullet$] ERM requires coordination across functions that are
  traditionally autonomous (finance, risk, business, strategy).

  \item[$\bullet$] Risk considerations often impose constraints on growth,
  leverage, or profitability, which creates natural resistance.

  \item[$\bullet$] The benefits of ERM are probabilistic and long-term,
  while the costs (foregone opportunities, additional controls) are immediate.
\end{itemize}

\begin{block}{Implementation insight}
ERM is difficult because it changes incentives and power, not because it
lacks tools.
\end{block}

\end{frame}

\begin{frame}{Common ERM failure modes}

Many organizations formally adopt ERM while neutralizing its impact.

\medskip

\begin{itemize}
  \item[$\bullet$] \textbf{Box-ticking ERM:} risk registers and policies exist,
  but they are disconnected from actual decisions.

  \item[$\bullet$] \textbf{Risk reporting without consequence:} dashboards are
  produced, but no action follows deterioration.

  \item[$\bullet$] \textbf{Over-measurement:} excessive metrics dilute attention
  and create a false sense of control.

  \item[$\bullet$] \textbf{Fragmentation:} risk categories are managed in silos,
  preventing portfolio-level insight.
\end{itemize}

\begin{block}{Diagnostic}
ERM fails when it produces information but does not alter behavior.
\end{block}

\end{frame}

\begin{frame}{Deeper failure modes: incentives and governance}

The most damaging ERM failures originate in incentive structures.

\medskip

\begin{itemize}
  \item[$\bullet$] Performance evaluation emphasizes short-term accounting
  outcomes rather than risk-adjusted or long-term value.

  \item[$\bullet$] Business units are rewarded for growth and volume, while
  risk functions bear responsibility without authority.

  \item[$\bullet$] Exception processes become systematic, undermining formal
  limits and risk appetite.

  \item[$\bullet$] Risk signals are softened or delayed as they move up the
  hierarchy.
\end{itemize}

\begin{block}{ERM reality}
Risk cannot be governed if those who take risk do not bear its consequences.
\end{block}

\end{frame}

\begin{frame}{ERM as a change management problem}

Implementing ERM requires altering how decisions are proposed, challenged,
and approved.

\medskip

\begin{itemize}
  \item[$\bullet$] Managers must justify decisions not only in terms of return,
  but also in terms of risk and resilience.

  \item[$\bullet$] ERM introduces structured challenge, which can be perceived
  as loss of autonomy or trust.

  \item[$\bullet$] Without visible support from top management, ERM is easily
  bypassed or marginalized.
\end{itemize}

\begin{block}{Key condition}
ERM succeeds only when senior leadership actively uses it in decisions.
\end{block}

\end{frame}

\begin{frame}{Integrating ERM into strategic planning}

ERM must shape strategic choices rather than react to them.

\medskip

\begin{itemize}
  \item[$\bullet$] Strategic plans should explicitly state key risk assumptions
  about markets, funding, regulation, and technology.

  \item[$\bullet$] Scenario analysis should test strategy robustness under
  adverse but plausible conditions.

  \item[$\bullet$] ERM should identify which strategic options are fragile
  versus resilient across scenarios.
\end{itemize}

\begin{block}{Strategic test}
Would this strategy still be acceptable if key assumptions fail?
\end{block}

\end{frame}


\begin{frame}{Integrating ERM into budgeting and forecasting}

Budgeting is a critical but often neglected ERM entry point.

\medskip

\begin{itemize}
  \item[$\bullet$] Budgets typically reflect optimistic point forecasts rather
  than distributions of outcomes.

  \item[$\bullet$] ERM introduces downside scenarios and stress assumptions
  into revenue, cost, and funding projections.

  \item[$\bullet$] Risk appetite should constrain budgeted growth, leverage,
  and concentration.
\end{itemize}

\begin{block}{Implementation rule}
If budgets ignore downside risk, ERM is not embedded.
\end{block}

\end{frame}

\begin{frame}{ERM in capital allocation and investment approval}

Capital allocation is where ERM becomes binding.

\medskip

\begin{itemize}
  \item[$\bullet$] Investment proposals should include risk-adjusted performance
  metrics and stress outcomes.

  \item[$\bullet$] ERM enables comparison of projects with different risk
  profiles competing for limited capital.

  \item[$\bullet$] Projects that are profitable in expectation may be rejected
  if they increase fragility at portfolio level.
\end{itemize}

\begin{block}{Decision logic}
ERM shifts the focus from isolated profitability to portfolio resilience.
\end{block}

\end{frame}

\begin{frame}{Dashboards: role and common pitfalls}

Dashboards are often overestimated as ERM solutions.

\medskip

\begin{itemize}
  \item[$\bullet$] Dashboards summarize information but do not create discipline
  by themselves.

  \item[$\bullet$] Excessive indicators dilute attention and mask deterioration.

  \item[$\bullet$] Backward-looking metrics provide comfort rather than early
  warning.
\end{itemize}

\begin{block}{Key insight}
Dashboards support ERM only when linked to escalation and action.
\end{block}

\end{frame}

\begin{frame}{Designing dashboards that support ERM}

Effective ERM dashboards are selective and forward-looking.

\medskip

\begin{itemize}
  \item[$\bullet$] Indicators should be explicitly linked to risk appetite
  and strategic objectives.

  \item[$\bullet$] Trends, concentrations, and stress metrics are more
  informative than point estimates.

  \item[$\bullet$] Visual design should make deterioration immediately visible,
  without interpretation.
\end{itemize}

\begin{block}{Design principle}
If escalation thresholds are not obvious, the dashboard fails.
\end{block}

\end{frame}

\begin{frame}{Reporting, escalation, and accountability}

Information without accountability does not reduce risk.

\medskip

\begin{itemize}
  \item[$\bullet$] ERM reporting should define clear thresholds that trigger
  mandatory discussion or action.

  \item[$\bullet$] Responsibilities for response must be explicit and
  documented.

  \item[$\bullet$] Failure to escalate should itself be treated as a risk event.
\end{itemize}

\begin{block}{ERM mechanism}
ERM works when risk signals force decisions, not explanations.
\end{block}

\end{frame}

\begin{frame}{What makes ERM work in practice}

\begin{itemize}
  \item[$\bullet$] ERM must be embedded in planning, budgeting, and capital
  allocation.

  \item[$\bullet$] Incentives and authority must be aligned with risk ownership.

  \item[$\bullet$] Tools and dashboards support judgment but do not replace it.
\end{itemize}

\begin{block}{Final takeaway}
ERM succeeds when it systematically changes how decisions are made under
uncertainty.
\end{block}

\end{frame}

% ========= PART IV — ERM across industries (2h): mini cases (Beamer) =========
% Assumes you already have \documentclass{beamer} and usual packages.
% Uses \item[$\bullet$] as requested and sentence case frame titles.

% ---------- Intro ----------
\begin{frame}{ERM across industries: four mini cases}
\begin{itemize}
  \item[$\bullet$] Same ERM principles, different industry economics.
  \item[$\bullet$] Each case is designed for diagnosis: identify ERM design, what failed, and why.
  \item[$\bullet$] Focus on interactions: risk categories, instruments, governance, incentives.
\end{itemize}
\begin{block}{Group task}
For each case: (i) identify the dominant risks and their interaction, (ii) infer ERM architecture, (iii) explain failure modes, (iv) propose 1--2 ERM improvements.
\end{block}
\end{frame}

% ---------- Case 1: Financial institutions (SVB) ----------
\begin{frame}{Mini case: financial institution — Silicon Valley Bank (SVB)}
SVB’s business model was highly concentrated in technology startups and venture
capital clients. Deposit growth surged in the low-rate period, with a deposit base
dominated by large, uninsured corporate balances that were highly confidence-sensitive.
The bank invested a substantial share of excess deposits into long-duration fixed-income
securities (Treasuries and agency MBS). While credit risk was limited, the duration
exposure created substantial interest rate risk as rates rose in 2022--2023.

As market values declined, unrealized losses accumulated. Risk measurement and regulatory
reporting existed, but ERM did not translate these signals into binding decisions. Hedging
of interest rate risk was reduced rather than increased, partly to preserve short-term
earnings. Liquidity stress testing did not fully reflect the speed and concentration of
outflows possible in a networked depositor base. When concerns surfaced, withdrawals
accelerated rapidly and the interaction between interest rate losses, liquidity pressure,
and confidence dynamics resulted in a rapid bank run and failure.
\end{frame}

\begin{frame}{SVB: questions for diagnosis}
\begin{enumerate}
  \item Which risks were central (interest rate, liquidity, concentration, confidence)? How did they amplify each other?
  \item What ERM elements likely existed (limits, ALM, stress tests, governance) and where did they fail to affect decisions?
  \item Why can compliance and measurement coexist with failure? Identify the decision points that should have changed.
  \item What would an ERM redesign prioritize: deposit concentration governance, hedging policy, escalation, or incentives?
\end{enumerate}
\end{frame}

% ---------- Case 2: Energy & commodities (Metallgesellschaft) ----------
\begin{frame}{Mini case: energy and commodities — Metallgesellschaft (1993)}
Metallgesellschaft, via its U.S. subsidiary, offered long-term fixed-price contracts
for petroleum products, creating long-horizon exposure to commodity price changes.
To hedge, it built a large futures position in short-dated oil contracts and rolled
them over time. Economically, the hedge aimed to offset long-term price risk using a
dynamic futures strategy.

The design was vulnerable to liquidity stress. As prices moved against the futures
position, variation margin outflows surged. Even if the hedge could work economically
over the long run, it required short-run funding capacity to survive adverse moves.
Management underestimated the scale and persistence of margin calls and the governance
needed to manage liquidity and unwind risk. The interaction between market risk,
liquidity risk, and governance led to forced unwinds at losses, turning a hedging
program into an enterprise-level crisis.
\end{frame}

\begin{frame}{Metallgesellschaft: questions for diagnosis}
\begin{enumerate}
  \item Why can a hedge be economically rational yet operationally fragile? Identify the key mechanism.
  \item Which enterprise risks were created by the hedge (liquidity, basis, governance, model dependence)?
  \item What hedging assessment should ERM have required (stress margin funding, survival horizon, escalation)?
  \item Propose an ERM-compliant redesign: instruments (options vs futures), limits, funding buffer, governance.
\end{enumerate}
\end{frame}

% ---------- Case 3: Non-financial corporation (Boeing 737 MAX) ----------
\begin{frame}{Mini case: non-financial corporation — Boeing and the 737 MAX}
Boeing operates in a safety-critical industry where operational and reputational
risks dominate financial hedging logic. The development of the 737 MAX took place
under intense competitive pressure and strong time-to-market constraints. Key design
and certification choices emphasized speed and cost containment. Safety risks related
to new flight control functionality were not escalated effectively, and internal
challenge mechanisms proved weak.

Formal risk processes and compliance structures existed, but they did not constrain
strategic and operational decisions. After two fatal accidents, the aircraft was
grounded globally, producing severe financial losses, reputational damage, and long
regulatory scrutiny. The episode illustrates how ERM can fail primarily through
governance, culture, and incentives, even when risks are knowable and procedures exist.
\end{frame}

\begin{frame}{Boeing: questions for diagnosis}
\begin{enumerate}
  \item What were the dominant risk families (operational safety, strategic, regulatory, reputational)?
  \item Where did ERM fail: identification, escalation, challenge, or incentive alignment?
  \item What signals should have triggered escalation, and why might they have been filtered or ignored?
  \item Propose two ERM architecture changes that increase challenge power without paralyzing innovation.
\end{enumerate}
\end{frame}

% ---------- Case 4: Public sector & infrastructure (megaprojects) ----------
\begin{frame}{Mini case: public sector and infrastructure — megaproject risk}
Large public infrastructure projects (rail, transit, airports) face long horizons,
multiple stakeholders, political oversight, and high public visibility. Initial project
approval often relies on optimistic forecasts for costs, timelines, and demand. Even
where formal risk frameworks exist, downside risks are systematically underestimated
because incentives favor launching projects and because accountability is diffuse.

As implementation proceeds, technical complexity, regulatory changes, procurement
frictions, and stakeholder opposition compound initial underestimation. Risk registers
may document risks, but ERM frequently lacks authority to halt projects, force major
redesign, or credibly escalate bad news. The result is a pattern of cost overruns,
delays, and legitimacy loss. The case highlights that in the public sector, the core
ERM challenge is often governance and incentive compatibility rather than measurement.
\end{frame}

\begin{frame}{Megaprojects: questions for diagnosis}
\begin{enumerate}
  \item Why do public projects tend to exhibit optimism bias and weak downside planning?
  \item How do political incentives and diffuse accountability constrain ERM effectiveness?
  \item What should an ERM escalation mechanism look like when decision rights are political?
  \item Propose an ERM design that introduces credible challenge (stage gates, independent review, stop rules).
\end{enumerate}
\end{frame}

% ---------- Cross-case synthesis ----------
\begin{frame}{Cross-case synthesis: what differs, what remains}
\begin{itemize}
  \item[$\bullet$] Financial institutions: small shocks can threaten solvency/liquidity; models and capital are central, but governance is decisive.
  \item[$\bullet$] Energy/commodities: derivatives are powerful, but margining and liquidity can dominate outcomes under stress.
  \item[$\bullet$] Non-financial corporations: ERM is mainly governance, escalation, and culture; technical risk is often not the binding constraint.
  \item[$\bullet$] Public sector: ERM is constrained by politics and accountability; challenge must be institutionalized through process and independence.
\end{itemize}
\begin{block}{Unifying insight}
ERM succeeds when it changes decisions under uncertainty and fits the organization’s incentives and constraints.
\end{block}
\end{frame}

% ---------- Optional: comparison matrix slide (compact) ----------
\begin{frame}{Comparison matrix: architecture, instruments, typical failure}
\small
\begin{tabular}{p{2.6cm}p{3.4cm}p{3.6cm}}
\hline
\textbf{Industry} & \textbf{Typical ERM anchor} & \textbf{Frequent failure mode} \\
\hline
Financial institutions & Capital, liquidity, stress tests, board risk governance & Model comfort + ignored signals; liquidity run dynamics \\
Energy \& commodities & Hedging policy + liquidity buffers + scenario culture & Margin spiral; basis/correlation breakdown; weak governance \\
Non-financial corporates & Governance, escalation, culture, controls & Incentives dominate; bad news filtered; safety/quality subordinated \\
Public sector \& infrastructure & Stage gates, independent review, accountability & Optimism bias; weak stop rules; political override \\
\hline
\end{tabular}

\medskip 

 \href{https://github.com/fsananou/bsb-erm/blob/main/case_studies.pdf}{\ovalbox{Practice}}

\end{frame}


\begin{frame}{Ten common myths about enterprise risk management}

\begin{itemize}
  \item[$\bullet$] ERM eliminates risk rather than managing risk-taking.
  \item[$\bullet$] More data and more models necessarily improve decisions.
  \item[$\bullet$] Risks can be fully understood through quantitative metrics.
  \item[$\bullet$] Hedging always reduces enterprise risk.
  \item[$\bullet$] ERM is primarily a compliance or regulatory requirement.
  \item[$\bullet$] Risk ownership can be delegated to a risk function.
  \item[$\bullet$] Scenario analysis is a substitute for judgment.
  \item[$\bullet$] Dashboards prevent surprises.
  \item[$\bullet$] A well-designed ERM system cannot fail.
  \item[$\bullet$] ERM failures are always due to lack of information.
\end{itemize}

\begin{block}{Final message}
ERM succeeds not by predicting the future, but by shaping decisions under
uncertainty.
\end{block}

\end{frame}

\end{document}