% !TEX TS-program = xelatex
% Compile with XeLaTeX or LuaLaTeX (because of fontspec)

\documentclass[10pt,oneside]{article}

% ------------------------------------------------
% Layout
% ------------------------------------------------
\usepackage[a4paper,margin=1in]{geometry}

% ------------------------------------------------
% Fonts (XeLaTeX/LuaLaTeX)
% ------------------------------------------------
\usepackage{fontspec}
\setmainfont{Carlito}

% ------------------------------------------------
% Typography & spacing
% ------------------------------------------------
\usepackage{microtype}
\usepackage{setspace}
\onehalfspacing
\setlength{\parindent}{0pt}
\setlength{\parskip}{0.6em}

% ------------------------------------------------
% Math
% ------------------------------------------------
\usepackage{amsmath,amssymb,mathtools}

% ------------------------------------------------
% Tables & lists
% ------------------------------------------------
\usepackage{booktabs}
\usepackage{enumitem}

% ------------------------------------------------
% Hyperlinks
% ------------------------------------------------
\usepackage[colorlinks=true,linkcolor=black,urlcolor=black,citecolor=black]{hyperref}

% ------------------------------------------------
% Title
% ------------------------------------------------
\title{\textbf{Risk Measurement Exercises: Value-at-Risk (VaR) and Expected Shortfall (ES)}}
\author{}
\date{}

% ------------------------------------------------
% Convenience macros (optional)
% ------------------------------------------------
\newcommand{\VaR}{\mathrm{VaR}}
\newcommand{\ES}{\mathrm{ES}}
\newcommand{\Loss}{L}
\newcommand{\R}{R}
\newcommand{\Normal}{\mathcal{N}}


\title{\textbf{Applied Case Studies}}
\author{Course: Enterprise Risk Management}
\date{}

\begin{document}
\maketitle

\onehalfspacing


\section{Case Study 1: Market Risk, Value-at-Risk and Risk Appetite}

\subsection{Context}

A mid-sized bank operates a trading desk managing a diversified market-risk
portfolio with a total market value of \textbf{€100 million}. The bank has
implemented an ERM framework centered on quantitative risk limits, with
Value-at-Risk (VaR) as the primary risk metric.

The desk is considered compliant as long as its estimated VaR remains below
the approved limit.

\subsection{Portfolio Composition}

The portfolio consists of the following exposures:
\begin{itemize}
    \item Equity exposure: €60 million invested in a broad equity index.
    \item Interest rate exposure: €40 million in a 10-year government bond,
    managed through duration exposure.
\end{itemize}

\subsection{Risk Parameters}

The following risk parameters are provided:
\begin{itemize}
    \item Daily equity volatility: 2\%.
    \item Daily volatility of the 10-year yield: 5 basis points.
    \item Correlation between equity returns and yield changes: 0.
    \item Interest rate sensitivity: a \textbf{+1 bp} increase in yields results
    in a \textbf{€0.20 million loss}.
\end{itemize}

Assume normally distributed risk factors and a one-day holding period.

\subsection{ERM Constraint}

The bank has set the following limit:
\begin{itemize}
    \item 1-day 99\% Value-at-Risk limit: \textbf{€3.0 million}.
\end{itemize}

\subsection{Questions}

\begin{enumerate}
    \item Compute the 1-day 99\% Value-at-Risk of the portfolio using a
    variance--covariance approach.
    \item Determine whether the trading desk complies with the VaR limit.
    \item Consider the following stress scenario:
    \begin{itemize}
        \item Equity index decline of 8\%.
        \item Increase of 30 basis points in the 10-year yield.
    \end{itemize}
    Compute the portfolio loss under this scenario.
    \item Explain why losses of this magnitude may occur despite compliance
    with the VaR limit.
    \item Identify at least two weaknesses of relying exclusively on VaR from
    an ERM perspective.
\end{enumerate}

\section{Case Study 2: Expected Shortfall, Governance and Escalation}

\subsection{Context}

Following internal reviews, the bank complements VaR with Expected Shortfall
(ES) reporting to better capture tail risk. ES is monitored daily and compared
against a predefined limit. Breaches are supposed to trigger immediate
escalation to senior management.

\subsection{Observed Portfolio Performance}

The daily profit-and-loss (P\&L) of the portfolio over the last 12 trading days
(in millions of euros) is reported below:

\begin{center}
\begin{tabular}{cc}
\toprule
Day & P\&L (€m) \\
\midrule
1  & +0.6 \\
2  & -1.1 \\
3  & +0.4 \\
4  & -2.3 \\
5  & +0.9 \\
6  & -0.8 \\
7  & -3.5 \\
8  & +0.5 \\
9  & -1.7 \\
10 & -4.2 \\
11 & -0.6 \\
12 & -6.8 \\
\bottomrule
\end{tabular}
\end{center}

\subsection{ERM Rules}

The ERM framework specifies:
\begin{itemize}
    \item 1-day 99\% Expected Shortfall limit: \textbf{€4.0 million}.
    \item Mandatory escalation in case of a breach.
\end{itemize}

\subsection{Questions}

\begin{enumerate}
    \item Sort the P\&L distribution from worst to best.
    \item Compute the empirical 1-day 99\% Expected Shortfall.
    \item Determine whether the ES limit is breached.
    \item The large loss observed on Day 12 occurred without prior escalation.
    Identify possible failures related to:
    \begin{itemize}
        \item Risk reporting,
        \item Governance architecture,
        \item Incentives or risk culture.
    \end{itemize}
    \item Explain why replacing VaR with Expected Shortfall does not, by itself,
    guarantee effective ERM.
\end{enumerate}



\section{Case study 3 : signals without escalation}

\subsection{Context}

A mid-sized European bank operates across corporate lending, structured finance,
and capital markets activities. The institution is considered well managed and
has experienced several years of stable profitability and moderate growth.

The bank is subject to prudential supervision and regularly undergoes internal
and external reviews of its risk management practices. No material weaknesses
have been identified in recent assessments.

\subsection{Risk governance practices}

Risk management within the bank is organized around a centralized risk function
that operates independently from business units. Risk policies define limits
for key risk categories and specify reporting obligations at different levels of
management.

Senior management receives periodic risk reports summarizing:
\begin{itemize}
    \item[$\bullet$] utilization of risk limits,
    \item[$\bullet$] stress-test outcomes,
    \item[$\bullet$] qualitative risk assessments prepared by risk functions.
\end{itemize}

A committee involving senior executives meets regularly to review these reports
and discuss emerging risks.

\subsection{Strategic development}

Over a two-year period, the bank expanded rapidly in structured finance and
specialized lending activities. Individual transactions complied with existing
risk limits and approval procedures.

However, several features characterized this expansion:
\begin{itemize}
    \item[$\bullet$] repeated concentration in similar counterparties and sectors,
    \item[$\bullet$] increasing reliance on favorable macroeconomic assumptions,
    \item[$\bullet$] frequent use of exceptions and temporary adjustments to
    internal limits.
\end{itemize}

While no formal breaches were recorded, internal risk assessments increasingly
highlighted rising portfolio sensitivity to adverse scenarios.

\subsection{Risk signals}

Risk analysts noted that portfolio-level exposures were approaching internally
defined thresholds under adverse conditions. These observations were included
in written risk reports and discussed during committee meetings.

The language used to describe these risks emphasized uncertainty rather than
immediate concern. Follow-up actions were left to business judgment, and no
explicit escalation was triggered.

\subsection{The event}

A sudden downturn affecting several correlated sectors led to a rapid
deterioration in asset quality and valuation losses. The combined impact
exceeded prior stress-test projections and resulted in a significant reduction
in capital buffers.

Although regulatory minimums were not immediately breached, the bank faced
heightened supervisory scrutiny, rating downgrades, and a loss of market
confidence.

\subsection{Questions}

\begin{enumerate}
    \item Based on the information provided, infer the \textbf{type of ERM
    framework} implemented by the bank. What elements support your inference?

    \item Reconstruct the bank’s \textbf{ERM architecture}. Who appears to:
    \begin{itemize}
        \item[$\bullet$] take risk,
        \item[$\bullet$] monitor and challenge risk,
        \item[$\bullet$] decide on escalation?
    \end{itemize}

    \item Identify the \textbf{early warning signals} that preceded the event.
    Why did they fail to translate into corrective action?

    \item Explain how incentives, governance dynamics, or risk culture may have
    weakened the effectiveness of the ERM system.

    \item Propose two concrete changes that would strengthen the bank’s ERM
    architecture and prevent similar failures.
\end{enumerate}

\section{Case study 4: hedging effectiveness under ERM (prospective vs retrospective)}

\subsection{Context}

A global industrial firm consumes a key commodity input. To stabilize margins,
the firm hedges its input cost exposure using standardized exchange-traded
futures linked to a benchmark commodity index.

Let $X$ denote the firm’s unhedged monthly profit-and-loss (P\&L) exposure
(driven mainly by commodity price changes). Let $H$ denote the monthly P\&L of
the hedging instrument (futures). The firm sets a hedge ratio $h$ and defines
the hedged P\&L as:
\[
Y = X - hH
\]

The goal is to reduce risk (variance and tail losses) \emph{without creating
unacceptable liquidity stress from margin calls}.

\subsection{Data provided}

Risk management estimates hedging parameters using a 24-month historical window
(\emph{normal period}). A market disruption then occurs (\emph{stress period}),
during which correlations and liquidity conditions change.

All statistics below are monthly, expressed in \textbf{€ million}.

\medskip

\begin{center}
\begin{tabular}{lcc}
\hline
 & Normal period (estimation) & Stress period (realized) \\
\hline
$\mathbb{E}[X]$ & 0.0 & -2.0 \\
$\mathbb{E}[H]$ & 0.0 & 0.0 \\
$\mathrm{Var}(X)$ & 100 & 225 \\
$\mathrm{Var}(H)$ & 64 & 144 \\
$\mathrm{Cov}(X,H)$ & 48 & 12 \\
\hline
\end{tabular}
\end{center}

\medskip

In the stress period, the firm also faces exchange margining:
\begin{itemize}
  \item[$\bullet$] Initial margin posted at inception: 10€m (funded from cash).
  \item[$\bullet$] Average monthly \emph{variation margin outflow} on the hedge:
  18€m.
\end{itemize}

Assume the firm has a liquidity buffer of 25€m dedicated to hedging operations.
If monthly margin outflows exceed this buffer, the firm must raise cash via
asset sales or emergency credit.

\subsection{Hedging rules used by the firm}

\begin{itemize}
  \item[$\bullet$] The hedge ratio is set using the minimum-variance benchmark:
  \[
  h^* = \frac{\mathrm{Cov}(X,H)}{\mathrm{Var}(H)}
  \]
  computed from the normal-period estimates.

  \item[$\bullet$] Hedging effectiveness is assessed via variance reduction:
  \[
  \text{HE} = 1 - \frac{\mathrm{Var}(Y)}{\mathrm{Var}(X)}
  \]

  \item[$\bullet$] Risk management reports prospective (ex ante) effectiveness
  using normal-period parameters, then evaluates retrospective (ex post)
  effectiveness using realized stress-period parameters \emph{while keeping the
  same hedge ratio $h$}.
\end{itemize}

\subsection{Tasks}

\begin{enumerate}
  \item \textbf{Hedge ratio.} Compute the minimum-variance hedge ratio $h^*$
  using the normal-period statistics.

  \item \textbf{Prospective effectiveness (ex ante).}
  Using $h=h^*$ and the normal-period statistics, compute:
  \[
  \mathrm{Var}(Y) = \mathrm{Var}(X) + h^2\mathrm{Var}(H) - 2h\,\mathrm{Cov}(X,H)
  \]
  and then compute $\text{HE}_{\text{pros}}$.

  \item \textbf{Retrospective effectiveness (ex post).}
  Using the \emph{same} hedge ratio $h=h^*$ but stress-period statistics,
  compute $\mathrm{Var}(Y)$ and $\text{HE}_{\text{retro}}$.

  \item \textbf{Interpretation.}
  Explain why $\text{HE}_{\text{retro}}$ differs from $\text{HE}_{\text{pros}}$
  in terms of correlation, basis risk, and regime change.

  \item \textbf{Liquidity assessment.}
  Compare the average monthly variation margin outflow (18€m) to the hedging
  liquidity buffer (25€m). Is the buffer sufficient on average?

  \item \textbf{ERM stress question.}
  Suppose variation margin outflows can spike to 40€m in bad months.
  Under ERM, list two governance or design changes that could reduce the
  probability of forced asset sales.

  \item \textbf{ERM integration.}
  Identify at least three risk categories that interacted in this episode and
  explain how ERM should have governed the interactions (not the silos).
\end{enumerate}


\section{Case study 5: when hedging increased risk}

\subsection{Context}

A large industrial firm operates globally and is highly exposed to commodity
price fluctuations. Its core activity relies on a key input whose market price
is volatile and subject to geopolitical and macroeconomic shocks.

To stabilize earnings and protect margins, management has historically relied
on derivative contracts to hedge price risk. The firm is considered
sophisticated and has a dedicated risk management function.

Over recent years, profitability has been stable and volatility of reported
earnings has declined.

\subsection{Risk governance and hedging strategy}

The firm’s risk management policy allows the use of derivative instruments
to hedge commodity price exposure over a rolling horizon.

\begin{itemize}
  \item[$\bullet$] Hedging is conducted primarily using futures and swap
  contracts linked to benchmark commodity prices.

  \item[$\bullet$] Hedge ratios are determined based on historical correlations
  between input costs and output prices.

  \item[$\bullet$] Hedging effectiveness is assessed ex ante using variance
  reduction metrics and reviewed periodically.
\end{itemize}

Risk reports show that the hedging program significantly reduces short-term
price volatility under normal market conditions.

\subsection{Quantitative information}

Let $X$ denote the unhedged price exposure of the firm and $H$ the payoff of
the hedging instrument.

The hedged position is defined as:
\[
Y = X - hH
\]

Based on historical data, the estimated minimum-variance hedge ratio is:
\[
h^* = \frac{\mathrm{Cov}(X,H)}{\mathrm{Var}(H)} = 0.9
\]

Prospective hedging effectiveness is estimated as:
\[
\text{HE}_{\text{pros}} = 1 - \frac{\mathrm{Var}(Y)}{\mathrm{Var}(X)} = 75\%
\]

As a result, management authorizes a large-scale hedging program aligned with
this ratio.

\subsection{The shock}

Following a sudden market disruption:
\begin{itemize}
  \item[$\bullet$] Correlations between the firm’s input prices and benchmark
  contracts weaken significantly.
  \item[$\bullet$] Futures markets experience sharp volatility and liquidity
  deteriorates.
  \item[$\bullet$] Margin requirements on derivative positions increase rapidly.
\end{itemize}

Although the underlying commodity price moves in the expected direction,
the firm faces substantial cash outflows due to margin calls.

\subsection{Outcomes}

Despite being economically hedged in the long term, the firm experiences:
\begin{itemize}
  \item[$\bullet$] severe short-term liquidity stress,
  \item[$\bullet$] forced asset sales to meet margin calls,
  \item[$\bullet$] a deterioration of credit metrics and investor confidence.
\end{itemize}

Retrospective hedging effectiveness falls sharply:
\[
\text{HE}_{\text{retro}} < 0
\]

Senior management questions how a risk-reducing strategy could generate such
outcomes.

\subsection{Questions}

\begin{enumerate}
  \item Identify the main risks faced by the firm. Which risk categories
  interacted in this episode?

  \item From an ERM perspective, explain why the hedging strategy appeared
  effective ex ante.

  \item Interpret the hedge ratio $h^*$. What assumptions underlie its use?

  \item Why did prospective hedging effectiveness fail to predict actual
  outcomes?

  \item Using the ERM framework, identify which secondary risks were
  insufficiently considered.

  \item How should ERM have governed the use of derivatives differently?

  \item Propose two changes to the firm’s ERM architecture or decision process
  that would have reduced the severity of the crisis.
\end{enumerate}



\end{document}
