% !TEX TS-program = xelatex
% Compile with XeLaTeX or LuaLaTeX (because of fontspec)

\documentclass[11pt,oneside]{article}

% ------------------------------------------------
% Layout
% ------------------------------------------------
\usepackage[a4paper,margin=1in]{geometry}

% ------------------------------------------------
% Fonts (XeLaTeX/LuaLaTeX)
% ------------------------------------------------
\usepackage{fontspec}
\setmainfont{Carlito}

% ------------------------------------------------
% Typography & spacing
% ------------------------------------------------
\usepackage{microtype}
\usepackage{setspace}
\onehalfspacing
\setlength{\parindent}{0pt}
\setlength{\parskip}{0.6em}

% ------------------------------------------------
% Math
% ------------------------------------------------
\usepackage{amsmath,amssymb,mathtools}

% ------------------------------------------------
% Tables & lists
% ------------------------------------------------
\usepackage{booktabs}
\usepackage{enumitem}

% ------------------------------------------------
% Hyperlinks
% ------------------------------------------------
\usepackage[colorlinks=true,linkcolor=black,urlcolor=black,citecolor=black]{hyperref}

% ------------------------------------------------
% Title
% ------------------------------------------------
\title{\textbf{Risk Measurement Exercises: Value-at-Risk (VaR) and Expected Shortfall (ES)}}
\author{}
\date{}

% ------------------------------------------------
% Convenience macros (optional)
% ------------------------------------------------
\newcommand{\VaR}{\mathrm{VaR}}
\newcommand{\ES}{\mathrm{ES}}
\newcommand{\Loss}{L}
\newcommand{\R}{R}
\newcommand{\Normal}{\mathcal{N}}


\title{\textbf{Enterprise Risk Management: Applied Case Studies}}
\author{Course: Enterprise Risk Management}
\date{}

\begin{document}
\maketitle

\onehalfspacing

\section{Introduction}

Enterprise Risk Management (ERM) is often presented as a comprehensive
framework combining governance, quantitative tools, and strategic
decision-making. In practice, many organizations formally comply with ERM
requirements while still experiencing severe losses or strategic failures.

The objective of the following case studies is to examine situations in which
ERM frameworks and quantitative risk metrics were in place, yet proved
insufficient. Students are asked to compute standard risk measures and then
analyze why ERM failed to prevent losses.

\section{Case Study 1: Market Risk, Value-at-Risk and Risk Appetite}

\subsection{Context}

A mid-sized bank operates a trading desk managing a diversified market-risk
portfolio with a total market value of \textbf{€100 million}. The bank has
implemented an ERM framework centered on quantitative risk limits, with
Value-at-Risk (VaR) as the primary risk metric.

The desk is considered compliant as long as its estimated VaR remains below
the approved limit.

\subsection{Portfolio Composition}

The portfolio consists of the following exposures:
\begin{itemize}
    \item Equity exposure: €60 million invested in a broad equity index.
    \item Interest rate exposure: €40 million in a 10-year government bond,
    managed through duration exposure.
\end{itemize}

\subsection{Risk Parameters}

The following risk parameters are provided:
\begin{itemize}
    \item Daily equity volatility: 2\%.
    \item Daily volatility of the 10-year yield: 5 basis points.
    \item Correlation between equity returns and yield changes: 0.
    \item Interest rate sensitivity: a \textbf{+1 bp} increase in yields results
    in a \textbf{€0.20 million loss}.
\end{itemize}

Assume normally distributed risk factors and a one-day holding period.

\subsection{ERM Constraint}

The bank has set the following limit:
\begin{itemize}
    \item 1-day 99\% Value-at-Risk limit: \textbf{€3.0 million}.
\end{itemize}

\subsection{Questions}

\begin{enumerate}
    \item Compute the 1-day 99\% Value-at-Risk of the portfolio using a
    variance--covariance approach.
    \item Determine whether the trading desk complies with the VaR limit.
    \item Consider the following stress scenario:
    \begin{itemize}
        \item Equity index decline of 8\%.
        \item Increase of 30 basis points in the 10-year yield.
    \end{itemize}
    Compute the portfolio loss under this scenario.
    \item Explain why losses of this magnitude may occur despite compliance
    with the VaR limit.
    \item Identify at least two weaknesses of relying exclusively on VaR from
    an ERM perspective.
\end{enumerate}

\section{Case Study 2: Expected Shortfall, Governance and Escalation}

\subsection{Context}

Following internal reviews, the bank complements VaR with Expected Shortfall
(ES) reporting to better capture tail risk. ES is monitored daily and compared
against a predefined limit. Breaches are supposed to trigger immediate
escalation to senior management.

\subsection{Observed Portfolio Performance}

The daily profit-and-loss (P\&L) of the portfolio over the last 12 trading days
(in millions of euros) is reported below:

\begin{center}
\begin{tabular}{cc}
\toprule
Day & P\&L (€m) \\
\midrule
1  & +0.6 \\
2  & -1.1 \\
3  & +0.4 \\
4  & -2.3 \\
5  & +0.9 \\
6  & -0.8 \\
7  & -3.5 \\
8  & +0.5 \\
9  & -1.7 \\
10 & -4.2 \\
11 & -0.6 \\
12 & -6.8 \\
\bottomrule
\end{tabular}
\end{center}

\subsection{ERM Rules}

The ERM framework specifies:
\begin{itemize}
    \item 1-day 99\% Expected Shortfall limit: \textbf{€4.0 million}.
    \item Mandatory escalation in case of a breach.
\end{itemize}

\subsection{Questions}

\begin{enumerate}
    \item Sort the P\&L distribution from worst to best.
    \item Compute the empirical 1-day 99\% Expected Shortfall.
    \item Determine whether the ES limit is breached.
    \item The large loss observed on Day 12 occurred without prior escalation.
    Identify possible failures related to:
    \begin{itemize}
        \item Risk reporting,
        \item Governance architecture,
        \item Incentives or risk culture.
    \end{itemize}
    \item Explain why replacing VaR with Expected Shortfall does not, by itself,
    guarantee effective ERM.
\end{enumerate}

\section{Conclusion}

These case studies illustrate that the presence of quantitative risk measures
and formal ERM limits does not ensure effective risk management. Understanding
why ERM fails requires analyzing not only models and metrics, but also risk
appetite definition, governance structures, escalation mechanisms, and
organizational incentives.

\end{document}
