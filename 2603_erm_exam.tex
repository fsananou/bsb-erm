% !TEX TS-program = xelatex
% Compile with XeLaTeX or LuaLaTeX (because of fontspec)

\documentclass[10pt,oneside]{article}

% ------------------------------------------------
% Layout
% ------------------------------------------------
\usepackage[a4paper,margin=1in]{geometry}

% ------------------------------------------------
% Fonts (XeLaTeX/LuaLaTeX)
% ------------------------------------------------
\usepackage{fontspec}
\setmainfont{Carlito}

% ------------------------------------------------
% Typography & spacing
% ------------------------------------------------
\usepackage{microtype}
\usepackage{setspace}
\onehalfspacing
\setlength{\parindent}{0pt}
\setlength{\parskip}{0.6em}

% ------------------------------------------------
% Math
% ------------------------------------------------
\usepackage{amsmath,amssymb,mathtools}

% ------------------------------------------------
% Tables & lists
% ------------------------------------------------
\usepackage{booktabs}
\usepackage{enumitem}

% ------------------------------------------------
% Hyperlinks
% ------------------------------------------------
\usepackage[colorlinks=true,linkcolor=black,urlcolor=black,citecolor=black]{hyperref}

% ------------------------------------------------
% Title
% ------------------------------------------------
\title{\textbf{Risk Measurement Exercises: Value-at-Risk (VaR) and Expected Shortfall (ES)}}
\author{}
\date{}

% ------------------------------------------------
% Convenience macros (optional)
% ------------------------------------------------
\newcommand{\VaR}{\mathrm{VaR}}
\newcommand{\ES}{\mathrm{ES}}
\newcommand{\Loss}{L}
\newcommand{\R}{R}
\newcommand{\Normal}{\mathcal{N}}


\title{\textbf{Final Exam}}
\author{Course: Enterprise Risk Management}
\date{}

\begin{document}
\maketitle
% ===================== ERM FINAL EXAM (LaTeX) =====================
% Paste into an article document (or adapt to your template).
% Designed to be challenging but fair, with computation exercises.
% ==================================================================

\subsection*{Instructions}
\begin{itemize}
  \item[$\bullet$] Duration: \textbf{3 hours}. All questions must be answered.
  \item[$\bullet$] Calculators are allowed. Show your steps for computations.
  \item[$\bullet$] Clarity of reasoning matters more than length.
  \item[$\bullet$] Unless explicitly asked, keep answers concise but rigorous.
\end{itemize}

\subsection*{Structure and grading}
\begin{itemize}
  \item[$\bullet$] Section I --- Risk concepts \& ERM foundations: \textbf{20 points}
  \item[$\bullet$] Section II --- Frameworks, architecture \& governance: \textbf{25 points}
  \item[$\bullet$] Section III --- Measurement, hedging \& decision-making (with computations): \textbf{30 points}
  \item[$\bullet$] Section IV --- Industry cases \& ERM failures: \textbf{25 points}
\end{itemize}

\vspace{2mm}
\hrule
\vspace{2mm}

% ===================== SECTION I =====================
\section*{Section I --- Risk concepts \& ERM foundations (20 points)}

\subsection*{Question 1 (10 points): risk vs uncertainty}
Explain the difference between \emph{risk} and \emph{uncertainty} in the context of ERM.
Why is this distinction central to the design of enterprise risk management systems?
Your answer must address (i) what can and cannot be reliably measured, (ii) the role of scenarios
versus point forecasts, and (iii) why judgment remains necessary even with quantitative tools.

\subsection*{Question 2 (10 points): why silo risk management fails}
Explain why managing risks in silos can increase enterprise-wide vulnerability, even if each risk
category appears well controlled locally. Provide a concrete example involving at least two risk
categories (e.g.\ market and liquidity; operational and reputational; credit and funding).

% ===================== SECTION II =====================
\section*{Section II --- ERM frameworks, architecture \& governance (25 points)}

\subsection*{Question 3 (10 points): frameworks are not ERM}
COSO ERM and ISO 31000 are widely used frameworks. Explain what these frameworks provide, but also
what they do not provide. Your answer must include:
\begin{itemize}
  \item[$\bullet$] what frameworks do well (structure, language, process discipline),
  \item[$\bullet$] how ``box-ticking ERM'' emerges,
  \item[$\bullet$] why governance and incentives dominate formal adoption.
\end{itemize}

\subsection*{Question 4 (15 points): architecture that makes risk information decision-relevant}
Consider the statement:
\begin{quote}
``ERM fails when risk information exists but does not change decisions.''
\end{quote}
Using concepts from the course, propose an ERM architecture that increases the probability that risk
signals will affect decisions. Your answer must discuss:
\begin{itemize}
  \item[$\bullet$] risk ownership vs.\ risk oversight,
  \item[$\bullet$] escalation thresholds and exception governance,
  \item[$\bullet$] the role of the board and senior management,
  \item[$\bullet$] why independence of the risk function is necessary but not sufficient.
\end{itemize}

% ===================== SECTION III =====================
\section*{Section III --- Measurement, hedging \& decision-making (30 points)}

\subsection*{Question 5 (10 points): RAROC and EVA (concept + computation)}

A business unit proposes a new portfolio with the following annual figures (in \texteuro{} million).
Expected net income (after expected losses) is $NI = 28$.
Economic capital allocated to absorb unexpected losses at the chosen confidence level is $EC = 200$.
The firm-wide hurdle rate (cost of capital) is $k = 11\%$.
Assume invested capital for EVA purposes is equal to economic capital for this exercise.

\begin{enumerate}
  \item[(a)] Define \textbf{RAROC} conceptually and compute it for this portfolio. \hfill (3 pts)
  \item[(b)] Define \textbf{EVA} conceptually and compute it. \hfill (3 pts)
  \item[(c)] A manager concludes: ``RAROC $>$ hurdle rate $\Rightarrow$ always accept.''
  Explain why this rule can be misleading in ERM practice (give \emph{two} reasons). \hfill (4 pts)
\end{enumerate}

\subsection*{Question 6 (20 points): hedging assessment (prospective vs retrospective) + tail risk}

A firm hedges a monthly exposure using a derivative. Let $X$ denote the unhedged monthly P\&L
exposure (in \texteuro{} million), and $H$ the monthly P\&L of the hedging instrument.
The hedged position is:
\[
Y = X - hH
\]
The firm sets the hedge ratio using normal-period estimates, then keeps that $h$ during stress.

\paragraph{Data (monthly, \texteuro{} million)}
Normal period (estimation):
\[
\mathrm{Var}(X)=100,\quad \mathrm{Var}(H)=64,\quad \mathrm{Cov}(X,H)=48
\]
Stress period (realized):
\[
\mathrm{Var}(X)=225,\quad \mathrm{Var}(H)=144,\quad \mathrm{Cov}(X,H)=12
\]

Hedge effectiveness is measured by variance reduction:
\[
\text{HE} = 1-\frac{\mathrm{Var}(Y)}{\mathrm{Var}(X)}
\]
with
\[
\mathrm{Var}(Y)=\mathrm{Var}(X)+h^2\mathrm{Var}(H)-2h\,\mathrm{Cov}(X,H)
\]

\begin{enumerate}
  \item[(a)] Compute the minimum-variance hedge ratio in normal times:
  \[
  h^*=\frac{\mathrm{Cov}(X,H)}{\mathrm{Var}(H)}
  \]
  \hfill (4 pts)

  \item[(b)] Using $h=h^*$, compute $\mathrm{Var}(Y)$ and $\text{HE}_{\text{pros}}$ in the normal period. \hfill (5 pts)

  \item[(c)] Keeping $h=h^*$, compute $\mathrm{Var}(Y)$ and $\text{HE}_{\text{retro}}$ using stress-period statistics. \hfill (5 pts)

  \item[(d)] Interpret the change from $\text{HE}_{\text{pros}}$ to $\text{HE}_{\text{retro}}$.
  Your explanation must mention \emph{regime change} and at least one of:
  basis/correlation breakdown, model risk, liquidity/market functioning. \hfill (3 pts)

  \item[(e)] Tail-risk check (discrete ES). In stress months, the unhedged loss distribution of $X$
  (in \texteuro{} million) is approximated by the following \emph{loss outcomes} (positive numbers are losses):
  \[
  L_X \in \{0,\,5,\,10,\,25,\,40\} \quad \text{with probabilities} \quad \{0.40,\,0.25,\,0.20,\,0.10,\,0.05\}
  \]
  (i) Compute $VaR_{95\%}(L_X)$. \hfill (1 pt)\\
  (ii) Compute $ES_{95\%}(L_X)$ (expected loss conditional on being in the worst 5\%). \hfill (2 pts)
\end{enumerate}

\paragraph{Short ERM question (no computation, included in Q6 marking)}
Explain why a derivative hedge that improves variance and even improves tail risk may still be
unacceptable from an ERM perspective. Your answer must identify at least \emph{two} ``secondary risks''
introduced by hedging instruments (e.g.\ liquidity/margining, counterparty, legal, basis). \hfill (Bonus within Q6, up to 2 pts)

% ===================== SECTION IV =====================
\section*{Section IV --- Industry cases \& ERM failures (25 points)}

\subsection*{Question 7 (12 points): cross-industry comparison}
Compare ERM in \emph{two} industries studied in the course (choose two):
financial institutions, energy \& commodities, non-financial corporations, public sector \& infrastructure.
Your answer must explain:
\begin{itemize}
  \item[$\bullet$] how industry economics shape ERM priorities,
  \item[$\bullet$] typical instruments and constraints (e.g.\ models/capital, hedging/liquidity buffers, controls/culture),
  \item[$\bullet$] why ERM tends to fail differently across these industries.
\end{itemize}

\subsection*{Question 8 (13 points): case-based diagnosis --- ``ERM existed, crisis still happened''}
A large firm has adopted a recognized ERM framework, produces regular risk reports, and uses quantitative
risk metrics. Despite this, it experiences a major crisis triggered by a combination of market stress,
liquidity pressure, and delayed escalation.

\medskip

Explain why ERM may have failed and identify the most likely failure modes. Your answer must:
\begin{itemize}
  \item[$\bullet$] distinguish measurement failure from governance failure,
  \item[$\bullet$] discuss incentives and escalation,
  \item[$\bullet$] explain why ``more data'' would not necessarily have prevented the crisis,
  \item[$\bullet$] propose \emph{two} concrete architecture or process changes that would reduce recurrence.
\end{itemize}

\vspace{2mm}
\hrule
\vspace{2mm}

\subsection*{End of exam}


\end{document}